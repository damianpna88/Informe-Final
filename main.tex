\documentclass[a4paper,openany,12pt]{book}
\usepackage[papersize={210mm,297mm},tmargin=25mm,bmargin=25mm,lmargin=25mm,rmargin=25mm]{geometry} 
\usepackage[spanish, mexico]{babel}
\usepackage[utf8]{inputenc}
\usepackage{mathtools}
\usepackage[breaklinks=true]{hyperref}
\usepackage{graphicx}
\usepackage{float}
\usepackage{seqsplit}
\usepackage{import}
\usepackage{array}
\usepackage{tabularx,ragged2e,booktabs,caption} % Para las tablas y ajuste a la linea
\usepackage{xr} % Referenciar secciones en otro tex
\usepackage{booktabs}
\usepackage{varwidth}

\usepackage{emptypage}

\usepackage{natbib}
\usepackage{todonotes}
\usepackage{blindtext}
\usepackage{hyperref}
\usepackage{caption}
\usepackage{enumerate}
\usepackage{listings}
\usepackage[T1]{fontenc}
\usepackage{minted}
%\externaldocument[anx-]{8-anexo/anexo}
\usepackage{array}
\newcolumntype{L}[1]{>{\raggedright\let\newline\\\arraybackslash\hspace{0pt}}m{#1}}
\newcolumntype{C}[1]{>{\centering\let\newline\\\arraybackslash\hspace{0pt}}m{#1}}
\newcolumntype{R}[1]{>{\raggedleft\let\newline\\\arraybackslash\hspace{0pt}}m{#1}}
\newsavebox\myv
% Para que no separe en silabas y mantenga justificado.
\tolerance=1
\emergencystretch=\maxdimen
\hyphenpenalty=10000
\hbadness=10000

% Pone el espacio entre párrafos
\setlength{\parskip}{6mm}

%Insertar paginas en Blanco
\usepackage{afterpage}
%Tablas largas
\usepackage{longtable} % para tablas largas

%Apendice
\usepackage{appendix}

% ENCABEZADOS Y PIES DE PAGINAS
\usepackage{fancyhdr}
\pagestyle{fancy}
\setlength{\headheight}{15pt}%Ancho del encabezado

% Numeración de Páginas
\lhead[]{}
\chead[]{}
\rhead[]{}
\cfoot[]{}
\fancyhead[LE,RO]{ \leftmark}% Aca hay una magia con el leftmark para q ponga los encabezados, si pones rightmark se manda otra...
\renewcommand{\headrulewidth}{0pt}% Ancho de la linea del encabezado
\fancyfoot[LO,RE]{\thepage}% Num de pagina alternante con paginas pares e impares

\fancypagestyle{plain}{% Formato para nuevos capitulos, incluye la bibliografia
\fancyhf{} 
\fancyfoot[LO,RE]{\thepage}
\renewcommand{\headrulewidth}{0pt}
\renewcommand{\footrulewidth}{0pt}}

%fix captionof error
\usepackage{hypcap}
\capstarttrue


\begin{document}
\import{caratula/}{caratula.tex}

\afterpage{\null\newpage}
\newpage

\frontmatter

%Indice General
\tableofcontents
%Indice de tablas
\listoftables
%Indice de Figuras
\listoffigures


\pagestyle{plain}
\import{Prefacio/}{Prefacio.tex}

\import{resumen/}{resumen.tex}
\afterpage{\null\newpage}
\newpage


%------------------------ACOMODO ESTILOS!------------------------------------
\pagestyle{fancy}
\setlength{\headheight}{15pt}%Ancho del encabezado

% Numeración de Páginas
\lhead[]{}
\chead[]{}
\rhead[]{}
\cfoot[]{}
\fancyhead[LO,RE]{ \leftmark}% Aca hay una magia con el leftmark para q ponga los encabezados, si pones rightmark se manda otra...
\renewcommand{\headrulewidth}{0pt}% Ancho de la linea del encabezado
\fancyfoot[LO,RE]{\thepage}% Num de pagina alternante con paginas pares e impares

\fancypagestyle{plain}{% Formato para nuevos capitulos, incluye la bibliografia
\fancyhf{} 
\fancyfoot[LO,RE]{\thepage}
\renewcommand{\headrulewidth}{0pt}
\renewcommand{\footrulewidth}{0pt}}
%------------------------------------------------------------------------------


\mainmatter

\import{introduccion/}{introduccion.tex}

\import{analisis_tematico/}{analisis_tematico.tex}

\import{requerimientos/}{requerimientos.tex}

\import{hardware/}{hardware.tex}

\import{Interfaz_Hard-Soft/}{interfazHardSoft.tex}

\import{software/}{software.tex}

\import{Pruebas/}{pruebas.tex}

\import{Conclusiones/}{Conclusiones.tex}

\fancyhead[LO,RE]{ANEXOS}

\import{Anexo/}{anexo.tex}

\fancyhead[LO,RE]{Bibliografía}


%Bibliografía

\nocite{*}
\bibliographystyle{apalike}
\bibliography{Bibliografia/biblio}

\end{document}


