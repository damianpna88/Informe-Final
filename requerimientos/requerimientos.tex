\chapter{Ingeniería de requerimientos}
\section{Recopilación de requerimientos}
    \par
    Los requerimientos para un sistema son la descripción detallada de los servicios proporcionados por el sistema y de sus restricciones operativas, \cite{Som05}. En lo que respecta al proceso de recopilación de requerimientos, el mismo consiste en identificar e interpretar las necesidades de los interesados y afectados por el proyecto en pos de facilitar el análisis de los mismos.
    
    \subsection{Interesados}
    \begin{itemize}
        \item Productores de Cerveza Artesanal: Usuarios finales del sistema, constatándose como los principales interesados.
        \item Alumnos: Son los responsables del desarrollo de la aplicación.
        \item Director y Co-Director: El Ing. Alejandro Fort Villa llevará a cabo la dirección del proyecto, mientras que el Ing. Roberto Nazareno se encargará de la codirección. Ambos profesionales desarrollan actividades de producción de cerveza artesanal.
    \end{itemize}
    
    \subsection{Técnicas utilizadas}
    \par
    Para la recopilación de requerimientos, fueron realizadas entrevistas con productores de cerveza artesanal, como Alejandro Fort Villa y Roberto Nazareno, entre otros. Durante las cuales, pudieron ser identificados distintos requerimientos que la aplicación debe cumplir tomando en consideración las necesidades al momento de realizar una producción. 
    \par
    Por otro lado, fueron analizadas funcionalidades de otras aplicaciones disponibles en el mercado como \textit{Beer Smith}\textsuperscript{\textregistered}, \textit{Brew-o-Matic}\textsuperscript{\textregistered} y \textit{Brewer’s Friend}\textsuperscript{\textregistered}, dado que las mismas realizan funciones similares a la propuesta en el presente proyecto.
    
    \subsection{Definición de requerimientos}
    %\begin{minipage}{0.95\textwidth}
    \par
    Con la información obtenida en la etapa de recopilación de requerimientos, fue definido un conjunto de requisitos funcionales y no funcionales, los cuales establecen las principales características de la aplicación. Los mismos fueron divididos en tres módulos: el módulo de monitoreo, de planificación y el módulo historial.
    %\end{minipage}
    
  %\begin{minipage}{0.95\textwidth}
  \begin{table}[H]
   \begin{center}
   \begin{tabularx}{\textwidth}{| X | X | X |}
   \hline
   \multicolumn{3}{|c|}{\textbf{Requerimientos funcionales}} \\
   \hline
    ID & Nombre & Descripción \\
   \hline
   \hline
        RF001 & Sensores & El sistema debe poder obtener datos de temperatura y pH dentro del macerador, como así también de temperatura ambiente.
        \\\hline
        RF002 & Múltiples sensores de temperatura  & El sistema debe proveer información de múltiples sensores ubicados en diferentes puntos dentro del macerador.
        \\\hline
        RF003 & Monitorear Maceración & El sistema debe permitir visualizar las variables: temperatura; condiciones climáticas; pH; y Actividad enzimática; a partir de los datos obtenidos de los sensores.
        \\\hline
        RF004 & Realizar seguimiento de un experimento de Maceración & El sistema debe permitir realizar el seguimiento de un experimento de maceración, brindando alertas ante desvíos o momentos de intervención del productor. \\\hline
        RF005 &  Planificar experimentos de Maceración & El sistema debe permitir definir las temperaturas, pH, duración de los intervalos de Maceraciones Simples y Complejas,
        \\\hline
        
    \end{tabularx}
    \label{ReqFuncionales_Parte1}
    \end{center}
    %\end{minipage}
    \end{table}
    
    %\begin{minipage}{0.95\textwidth}
    \begin{table}[H]
    \begin{center}
    \begin{tabularx}{\textwidth}{| X | X | X |}
    \hline
    \multicolumn{3}{|c|}{\textbf{Requerimientos funcionales cont. 1}} \\
    \hline
    ID & Nombre & Descripción \\
    \hline
    \hline
         &   & características de los granos que se utilizarán y el volumen deseado para la misma.
        \\\hline
        RF006 & Alta y Baja de Maceraciones & El sistema debe permitir crear o eliminar una maceración y todos los experimentos relacionados a esta.  \\\hline
        RF007 & Almacenar / Eliminar experimentos  & El sistema debe permitir almacenar y eliminar experimentos de maceración.
        \\\hline
        RF008 & Calcular insumos para Maceración Planificada & A partir de una planificación de maceración, el sistema debe permitir calcular la cantidad de cada tipo de grano, y el volumen y temperatura del agua para cada intervalo de maceración compleja.
        \\\hline
        RF009 & Observar datos evolutivos de la maceración & El sistema debe proveer información que represente la evolución del proceso de maceración en curso a partir de las variables de temperatura y pH, y su acción combinada en la actividad enzimática. De forma adicional debe proveer datos de las condiciones ambientales en las que se está realizando la maceración.
        \\\hline
        RF010 & Observar gráficas históricas y comparativas entre experimentos de maceración & El sistema debe proveer gráficas de cada experimento de maceración y otras donde se evidencien sus correspondientes comparaciones. \\\hline
    \end{tabularx}
    \label{ReqFuncionales_Parte2}
    \end{center}
    %\end{minipage}
    \end{table}
    
    %\begin{minipage}{0.95\textwidth}
    \begin{table}[H]
    \begin{center}
    \begin{tabularx}{\textwidth}{| X | X | X |}
    \hline
    \multicolumn{3}{|c|}{\textbf{Requerimientos funcionales cont. 2}} \\
    \hline
    ID & Nombre & Descripción \\
    \hline
    \hline

         RF011 & Cálculo de rendimientos en base a datos estadísticos & El sistema debe realizar optimizaciones en base a estimaciones del rendimiento del grano a partir de datos obtenidos de los experimentos de maceración.
         \\\hline
         RF012 & Configurar métodos de obtención de medidas representativas de temperatura & El sistema debe permitir configurar la forma en que se establece la representatividad de las múltiples medidas de temperatura obtenidas por los diferentes sensores (media, promedio, etc).
     \\\hline
    \end{tabularx}
    \label{ReqFuncionales_Parte3}
    \end{center}
    %\end{minipage}
    \end{table}
    
    %\begin{minipage}{0.95\textwidth}
    \begin{table}[H]
    \begin{center}
    \begin{tabularx}{\textwidth}{| X | X | X |}
    \hline
    \multicolumn{3}{|c|}{\textbf{Requerimientos no funcionales}} \\
    \hline
    ID & Nombre & Descripción \\
    \hline
    \hline
         RNF001 & Usabilidad & El sistema debe contar con una interfaz intuitiva, con información útil y clara de leer. \\\hline
         RNF002 & Costo de hardware & Los insumos a ser utilizados para la construcción de este componente deben ser adquiridos con la intención de reducir el costo total al mínimo.
         \\\hline
         RNF003 & Elección de plataforma para software & La plataforma del componente de software debe tener amplio acceso en el mercado argentino. 
         \\\hline
         RNF004 & Demoras aceptables de transmisión entre componentes & La interfaz debe permitir la transmisión de mediciones de un experimento con una demora máxima establecida\footnote{Dada la duración promedio de 60 min. de los experimentos, se considera aceptable un valor no menor a 1 min. como máximo}, tal que esta cumpla la condición de no entorpecer la detección de desvíos del preparado.
         \\ \hline
     \end{tabularx}
    \label{ReqNoFuncionales_Parte2}
    \end{center}
    \end{table}
    %\end{minipage}
    
    %\begin{minipage}{0.95\textwidth}
    \begin{table}[H]
    \begin{center}
    \begin{tabularx}{\textwidth}{| X | X | X |}
    \hline
    \multicolumn{3}{|c|}{\textbf{Requerimientos no funcionales cont.}} \\
    \hline
    ID & Nombre & Descripción \\
    \hline
    \hline          

        RNF005 & Recuperación de datos ante desconexiones & El sistema debe ser capaz de almacenar en el componente de hardware una copia de seguridad de los datos de la experiencia que se esté realizando, de manera de no perder datos antes eventuales pérdidas de conexión.
        \\ \hline
        RNF006 & Tamaño reducido & El componente electrónico debe tener un tamaño reducido, de manera de no estorbar las actividades de producción.
         \\\hline
    \end{tabularx}
    \label{ReqNoFuncionales_Parte2}
    \end{center}
    %\end{minipage}
    \end{table}
    
    %\begin{minipage}{0.95\textwidth}
    
    \par
    A partir de estos requerimientos, se identificó el actor del sistema y las
    interacciones que deben realizar con el mismo:
    \par
    \textbf{Cervecero}: Es el usuario del sistema, quien se encarga de definir las planificaciones de maceración, utilizar el monitoreo de variables para dar seguimiento al proceso y realizar análisis a partir de las experiencias y la información obtenida de estas.
    %\end{minipage}
 
    %\begin{minipage}{0.95\textwidth}
    \subsection{Casos de uso}
    %\hfill \break
    %\hfill \break
    %esto sería el current values?
    \begin{table}[H]
    
    \begin{center}
    \begin{tabularx}{\textwidth}{ | X | X |}
        \hline
        \multicolumn{2}{|c|}{\textbf{Caso de Uso: Monitorear variables - CU001}} \\
        \hline
        \multicolumn{2}{|l|}{Actor: Productor de Cerveza} \\
        \hline
        Curso Normal & Curso Alternativo \\
        \hline
        1- El CU comienza cuando el usuario hace \textit{click} en el botón “Monitorear Variables”. & \\
        \hline
        2- El sistema muestra una pantalla con los valores de temperatura y pH en proceso de monitoreo. & 2.1- El sistema no detecta los sensores. 
        El sistema muestra el mensaje “No pudieron obtenerse los datos del componente de hardware. Verificar conexión “. Vuelve a la pantalla anterior.
        \\
        \hline
    \end{tabularx}
    \label{CU001a}
    \end{center}
    \end{table}
    %\end{minipage}    
        
    %\begin{minipage}{0.95\textwidth}
%    \begin{table}[H]
%    \begin{center}
%    \begin{tabularx}{\textwidth}{ | X | X |}
%        \hline
%        \multicolumn{2}{|c|}{\textbf{Caso de Uso: Monitorear variables - %CU001 cont.}} \\
%        \hline
%        Curso Normal & Curso Alternativo \\
%        \hline
%        
%        2- El sistema muestra una pantalla con los valores de temperatura y pH en proceso de monitoreo. & 2.1- El sistema no detecta los sensores. 
        %El sistema muestra el mensaje “No pudieron obtenerse los datos del componente de hardware. Verificar conexión “. Vuelve a la pantalla anterior.
        %\\
        %\hline
    %\end{tabularx}
    %\label{CU001b}
    %\end{center}
    %\end{minipage}
%\end{table}

%\begin{minipage}{0.95\textwidth}
    \begin{table}[H]
    %planning activity
    \begin{center}
    \begin{tabularx}{\textwidth}{ | X | X |}
        \hline
        \multicolumn{2}{|c|}{\textbf{Caso de Uso: Cargar nueva maceración - CU002}} \\
        \hline
        \multicolumn{2}{|l|}{Actor: Productor de Cerveza} \\
        \hline
        Curso Normal & Curso Alternativo \\
        \hline
        1- El CU comienza cuando el usuario hace \textit{click} en el botón “Nueva Maceración”. & \\
        \hline
        2- El sistema despliega un formulario con los datos necesarios para la realización de una maceración. &
        \\
        \hline
        3- El usuario ingresa el nombre identificativo de la maceración, el volumen de producción deseada, las características y porcentajes de la/s malta/s que se utilizarán, y el tipo de maceración que será realizada. Finalmente presiona el botón “Guardar”. &
        \\
        \hline
        4- El sistema emitirá un mensaje indicando que los datos ingresados han sido guardados exitosamente.  & 
        4.1- Ya existe este nombre en la base de datos. El sistema mostrará una alerta con el mensaje “Nombre ya utilizado, ingrese uno diferente”.\newline 4.2- Se utilizaron caracteres inválidos para rellenar los espacios. El sistema mostrará una alerta mencionando el dato inválido y una lista de los valores inválidos.
        \\
        \hline
    \end{tabularx}
    \label{CU002}
    \end{center}
    \end{table}
    %\end{minipage}
    
    %\begin{minipage}{0.95\textwidth}
    \begin{table}[H]
    %experiment activity
    \begin{center}
    \begin{tabularx}{\textwidth}{ | X | X |}
        \hline
        \multicolumn{2}{|c|}{\textbf{Caso de Uso: Gestionar maceración - CU003}} \\
        \hline
        \multicolumn{2}{|l|}{Actor: Productor de Cerveza} \\
        \hline
        Curso Normal & Curso Alternativo \\
        \hline
        1- El CU comienza cuando el usuario selecciona la maceración de la lista de maceraciones cargadas. & \\
        \hline
        2- El sistema despliega una pantalla con los experimentos realizados y una serie de botones para: eliminar maceración (CU009), eliminar experimento (CU008), iniciar nuevo experimento (CU007), ver datos cargados de la maceración (CU004), ver histórico de maceración (CU005 y CU006) & 2.1- No existen experimentos realizados, la lista se muestra vacía.
        \\
        \hline
    \end{tabularx}
    \label{CU003a}
    \end{center}
    %\end{minipage}
    \end{table}
   %\begin{minipage}{0.95\textwidth}
    
%    \begin{center}
%    \begin{tabularx}{\textwidth}{ | X | X |}
%        \hline
%        \multicolumn{2}{|c|}{\textbf{Caso de Uso: Gestionar maceración - % CU003 cont.}} \\
%        \hline
%        Curso Normal & Curso Alternativo \\
%        \hline    
%        
%    \end{tabularx}
%    \label{CU003b}
%    \end{center}

    %este sería el info mash, planning activity hardcodeado
    \begin{table}[H]
    \begin{center}
    \begin{tabularx}{\textwidth}{ | X | X |}
        \hline
        \multicolumn{2}{|c|}{\textbf{Caso de Uso: Ver datos cargados de la maceración - CU004}} \\
        \hline
        \multicolumn{2}{|l|}{Actor: Productor de Cerveza} \\
        \hline
        Curso Normal & Curso Alternativo \\
        \hline
        1- El CU comienza cuando el usuario hace \textit{click} en el botón “ver datos cargados de la maceración” en el panel "Gestionar Maceración" CU003. & \\
        \hline
        2- El sistema muestra el tipo de maceración, el volumen de empaste a usar, la densidad deseada, la cantidad de malta de cada tipo a utilizar en base teórica y el volumen y temperatura de agua requerido para cada etapa. & 2.1- En caso de haberse repetido más de tres experimentos con la maceración seleccionada, se mostrará la cantidad de insumos en base ajustada. \\
        \hline
        
    \end{tabularx}
    \label{CU004}
    \end{center}
    %\end{minipage}
    \end{table}
    
    %\begin{table}[H]
    %    \begin{center}
%    \begin{tabularx}{\textwidth}{ | X | X |}
%        \hline
%        \multicolumn{2}{|c|}{\textbf{Caso de Uso: Gestionar maceración - % CU003 cont.}} \\
%        \hline
%        Curso Normal & Curso Alternativo \\
%        \hline    
%        2- El sistema muestra el tipo de maceración, el volumen de empaste a usar, la densidad deseada, la cantidad de malta de cada tipo a utilizar en base teórica y el volumen y temperatura de agua requerido para cada etapa. & 2.1- En caso de haberse repetido más de tres experimentos con la maceración seleccionada, se mostrará la cantidad de insumos en base ajustada. \\
%        \hline
%    \end{tabularx}
%    \label{CU00b}
%    \end{center}
    
    %\begin{minipage}{0.95\textwidth}
    \begin{table}[H]
    %general fragment
    \begin{center}
    \begin{tabularx}{\textwidth}{ | X | X |}
        \hline
        \multicolumn{2}{|c|}{\textbf{Caso de Uso: Ver datos estadísticos históricos generales - CU005}} \\
        \hline
        \multicolumn{2}{|l|}{Actor: Productor de Cerveza} \\
        \hline
        Curso Normal & Curso Alternativo \\
        \hline
        1- El CU comienza cuando el usuario hace \textit{click} en el botón “Datos Históricos” en la pantalla de ``Gestión de maceración''. & 1.1 No existen experimentos realizados, el sistema emite un mensaje alertando la situación y cierra la pantalla.\\
        \hline
        2- El sistema muestra un botón con la etiqueta ``Cambiar el formato de los gráficos''& 2.1- El usuario presiona el botón, el sistema alterna el formato de los gráficos entre lineares y caja-bigote. \\
        \hline
        
        3- El sistema muestra gráficas temporales de temperatura promedio, pH, activación de enzimas, temperatura de todos los experimentos de la maceración. & \\
        \hline
        
        4- El sistema muestra rendimiento del equipo para esta maceración y la cantidad de insumos en base teórica, ajustada y práctica. & 4.1 En caso de no haber realizado mas de tres experimentos, el valor teórico y ajustado coinciden.\\
        \hline
    
    \end{tabularx}
    \label{CU005}
    \end{center}
    %\end{minipage}
    \end{table}
    
    %\begin{minipage}{0.95\textwidth}
    \begin{table}[H]
    \begin{center}
    \begin{tabularx}{\textwidth}{ | X | X |}
        \hline
        \multicolumn{2}{|c|}{\textbf{Caso de Uso: Ver histórico por experimento - CU006}} \\
        \hline
        \multicolumn{2}{|l|}{Actor: Productor de Cerveza} \\
        \hline
        Curso Normal & Curso Alternativo \\
        \hline
        1- El CU comienza cuando el usuario hace \textit{click} en el botón ”Histórico por experimento”. & \\
        \hline
        2- El sistema despliega gráficas de temperatura promedio o por sensor, de cada experimento, ordenadas en forma descendente por fecha. Precedidas por la fecha de cada Experimento. & 
        \\
        \hline
    \end{tabularx}
    \label{CU006}
    \end{center}
    %\end{minipage}
    \end{table}
    
    
    %\begin{minipage}{0.95\textwidth}
    \begin{table}[H]
    \begin{center}
    \begin{tabularx}{\textwidth}{ | X | X |}
        \hline
        \multicolumn{2}{|c|}{\textbf{Caso de Uso: Iniciar nuevo experimento de maceración - CU007}} \\
        \hline
        \multicolumn{2}{|l|}{Actor: Productor de Cerveza} \\
        \hline
        Curso Normal & Curso Alternativo \\
        \hline
        1- El CU comienza cuando el usuario elige la opción “iniciar experimento” en el panel "Gestionar Maceración" \ref{CU002}. & \\
        \hline
        2- El sistema despliega un panel donde el usuario podrá visualizar los valores recolectados, planificados y los desvíos entre estos respecto a temperatura y pH durante el proceso.  & 2.1- En caso de ocurrir un desvío mayor al tolerado, el sistema emite una notificación indicando la situación. \newline 2.2- En caso de ocurrir algún problema con alguno de los sensores, el sistema indica la situación.
        \\
        \hline
        3- El usuario selecciona la opción configurar temperatura. El sistema enseña un panel para configurar los sensores de temperatura activos y el método para promediar los valores de temperatura. & En caso que el usuario modifique la configuración, el sistema modifica la visualización de acuerdo a la misma\\ 
        \hline
        
        4- El sistema muestra en el panel información relacionada a la activación de enzimas, temperatura y humedad ambiental, el tiempo transcurrido, el porcentaje de avance en conjunto con la etapa actual de la maceración.& \\
        \hline
        
        5- El sistema muestra un panel con información relacionada a cada etapa dentro de la maceración en curso: Tiempo restante para el inicio de la misma y volumen y temperatura de agua en el caso de una maceración por infusión o la cantidad de empaste a retirar en el caso de una maceración por decocción. & \\
        \hline
        
        6- El sistema muestra las opciones ``Finalizar experimento'' y ``Cancelar experimento'' & 6.1- En caso que el usuario seleccione la opción ``Finalizar experimento'' y el experimento hay alcanzado el total de la cantidad de mediciones, el sistema muestra un panel para que el usuario ingrese el valor de densidad específica obtenida.\newline 6.2- En caso que el usuario seleccione la opción ``Finalizar experimento'' y el experimento no haya alcanzado la cantidad de mediciones a realizar, el sistema enseña un mensaje indicando que aún no se puede finalizar el experimento.\\ 
        \hline
        
        \end{tabularx}
    \label{CU007_a}
    \end{center}
    %\end{minipage}
    \end{table}
    
    %\begin{minipage}{0.95\textwidth}
    \begin{table}[H]
    \begin{center}
    \begin{tabularx}{\textwidth}{ | X | X |}
    \hline
        \multicolumn{2}{|c|}{\textbf{Caso de Uso: Monitorear experimento de maceración - CU007 cont.}} \\
        \hline
        %\multicolumn{2}{|l|}{Actor: Productor de Cerveza} \\
        %\hline
    
        Curso Normal & Curso Alternativo \\
        \hline
        
         & 6.3- En caso que el usuario seleccione la opción ``Cancelar experimento'', el sistema finalizará el experimento y volverá a la pantalla de ``Gestión de maceración" luego que el usuario indique que esta seguro que desea cancelar el experimento.\\
        \hline

    \end{tabularx}
    \label{CU007_b}
    \end{center}
    %\end{minipage}
    \end{table}
    
    %\begin{minipage}{0.95\textwidth}
    \begin{table}[H]
    \begin{center}
    \begin{tabularx}{\textwidth}{ | X | X |}
        \hline
        \multicolumn{2}{|c|}{\textbf{Caso de Uso: Eliminar maceración - CU008}} \\
        \hline
        \multicolumn{2}{|l|}{Actor: Productor de Cerveza} \\
        \hline
        Curso Normal & Curso Alternativo \\
        \hline
        1- El CU comienza cuando el usuario hace \textit{click} en el botón “Eliminar maceración” en el panel de Maceraciones. & \\
        \hline
        2- El sistema despliega un panel con un mensaje que indica que se perderán todos los datos relacionados a esa maceración y opciones para aceptar o cancelar. &
        \\
        \hline
        3- El usuario elige una de las opciones del panel. & 3.1- El usuario elige la opción “Aceptar”. El sistema elimina todos los datos y emite una notificación que los datos han sido eliminados correctamente.\newline
        3.2- El usuario elige la opción “Cancelar”. El sistema vuelve al menú “Ver Maceraciones”.
        \\
        \hline
    \end{tabularx}
    \label{CU008}
    \end{center}
    %\end{minipage}
    \end{table}
    
    %\begin{minipage}{0.95\textwidth}
    \begin{table}[H]
    \begin{center}
    \begin{tabularx}{\textwidth}{ | X | X |}
        \hline
        \multicolumn{2}{|c|}{\textbf{Caso de Uso: Eliminar experimento de maceración - CU009}} \\
        \hline
        \multicolumn{2}{|l|}{Actor: Productor de Cerveza} \\
        \hline
        Curso Normal & Curso Alternativo \\
        \hline
        1- El CU comienza cuando el usuario hace \textit{click} en el botón "Eliminar Experimentos" perteneciente al panel Histórico Completo. & \\
        \hline
        2- El sistema despliega una alerta con el mensaje “Está seguro que desea eliminar estos experimentos? (Los mismos serán eliminados de forma permanente)”. &
        \\
        \hline
        3- El usuario presiona el botón “Confirmar eliminado”. & 3.1- El usuario hace \textit{click} en el botón “Cancelar”.
        \\
        \hline
        4- El sistema elimina los experimentos y regresa al caso de uso “Ver Histórico Completo - CU005 y CU006”. & 4.1- El sistema vuelve al caso de uso “Gestionar maceración - CU003”.
        \\
        \hline
    \end{tabularx}
    \label{CU009}
    \end{center}
    %\end{minipage}
    \end{table}
    
    \subsubsection{Diagrama de Casos de uso}
    El mismo se encuentra presente en el Anexo, Figura \ref{DiagCU}.

	
    
    
    
    

    