\chapter{Ingeniería de Requerimientos}
\section{Recopilación de Requerimientos}
    \par
    Los requerimientos para un sistema son la descripción detallada de los servicios proporcionados por el sistema y de sus restricciones operativas (Sommerville, 2005). En lo que respecta al proceso de recopilación de requerimientos, el mismo consiste en identificar e interpretar las necesidades de los interesados y afectados por el proyecto en pos de facilitar el análisis de los mismos.
    
    \subsection{Interesados}
    \begin{itemize}
        \item Productores de Cerveza Artesanal: Usuario final del sistema, constatándose como el principal interesado.
        \item Alumnos: Son los responsables del desarrollo de la aplicación.
        \item Director y Co-Director: El Ing. Alejandro Fort Villa llevará a cabo la dirección del proyecto, mientras que el Ing. Roberto Nazareno se encargará de la codirección. Ambos profesionales desarrollan actividades de producción de cerveza artesanal.
    \end{itemize}
    
    \subsection{Técnicas utilizadas}
    \par
    Para la recopilación de requerimientos, fueron realizadas entrevistas con productores de cerveza artesanal, como Alejandro Fort Villa y Roberto Nazareno, entre otros. Durante las cuales, pudieron ser identificados distintos requerimientos que la aplicación debe cumplir tomando en consideración las necesidades al momento de realizar una producción. 
    \par
    Por otro lado, fueron analizadas funcionalidades de otras aplicaciones disponibles en el mercado como Beer Smith, Brew-o-Matic, Brewer’s Friend, dado que las mismas realizan funciones similares.
    
    \subsection{Definición de Requerimientos}
    \begin{minipage}{0.95\textwidth}
    \par
    Con la información obtenida en la etapa de recopilación de requerimientos, fue definido un conjunto de requisitos funcionales y no funcionales, los cuales establecen las principales características de la aplicación. Los mismos fueron divididos en tres módulos: el módulo de monitoreo, de planificación y el módulo historial.
    \end{minipage}
    
  \begin{minipage}{0.95\textwidth}
   \begin{center}
   \begin{tabularx}{\textwidth}{| X | X | X |}
   \hline
   \multicolumn{3}{|c|}{\textbf{Requerimientos funcionales}} \\
   \hline
    ID & Nombre & Descripción \\
   \hline
   \hline
        RF001 & Sensores & El sistema debe poder obtener datos de temperatura y pH dentro del macerador, como así también de temperatura ambiente.
        \\\hline
        RF002 & Múltiples sensores de temperatura  & El sistema debe proveer información de múltiples sensores ubicados en diferentes puntos dentro del macerador.
        \\\hline
        RF003 & Monitorear Maceración & El sistema debe permitir visualizar las variables: temperatura; condiciones climáticas; pH; y Actividad enzimática; a partir de los datos obtenidos de los sensores.
        \\\hline
        RF004 & Realizar seguimiento de un experimento de Maceración & El sistema debe permitir realizar el seguimiento de un experimento de maceración, brindando alertas ante desvíos o momentos de intervención del productor. \\\hline
        
    \end{tabularx}
    \label{ReqFuncionales_Parte1}
    \end{center}
    \end{minipage}
    
    \begin{minipage}{0.95\textwidth}
    \begin{center}
    \begin{tabularx}{\textwidth}{| X | X | X |}
    \hline
    \multicolumn{3}{|c|}{\textbf{Requerimientos funcionales cont. 1}} \\
    \hline
    ID & Nombre & Descripción \\
    \hline
    \hline
        RF005 &  Planificar experimentos de Maceración & El sistema debe permitir definir las temperaturas, pH, duración de los intervalos de Maceraciones Simples y Complejas, características de los granos que se utilizarán y el volumen deseado para la misma.
        \\\hline
        RF006 & Alta y Baja de Maceraciones & El sistema debe permitir crear, o eliminar una maceración, y todos los experimentos relacionados a esta.  \\\hline
        RF007 & Almacenar / Eliminar experimentos  & El sistema debe permitir almacenar datos inherentes al experimento de maceración y luego permitir eliminar el experimento.
        \\\hline
        RF008 & Calcular insumos para Maceración Planificada & El sistema debe permitir a partir de una planificación de maceración calcular la cantidad de cada tipo de grano, y el volumen y temperatura del agua para cada intervalo de maceración compleja.
        \\\hline
        RF009 & Observar datos evolutivos de la maceración & El sistema debe proveer información que represente la evolución del proceso de maceración en curso a partir de las variables de temperatura y pH, y su acción combinada en la actividad enzimática. De forma adicional debe proveer datos de las condiciones ambientales en las que se está realizando la maceración.
        \\\hline
    \end{tabularx}
    \label{ReqFuncionales_Parte2}
    \end{center}
    \end{minipage}
    
    \begin{minipage}{0.95\textwidth}
    \begin{center}
    \begin{tabularx}{\textwidth}{| X | X | X |}
    \hline
    \multicolumn{3}{|c|}{\textbf{Requerimientos funcionales cont. 2}} \\
    \hline
    ID & Nombre & Descripción \\
    \hline
    \hline
         RF010 & Observar gráficas históricas y comparativas entre experimentos de maceración & El sistema debe proveer gráficas de cada experimento de maceración y otras donde se evidencien sus correspondientes comparaciones. \\\hline
         RF011 & Cálculo de rendimientos en base a datos estadísticos & El sistema debe realizar optimizaciones en base a estimaciones del rendimiento del grano a partir de datos obtenidos de los experimentos de maceración.
         \\\hline
         RF012 & Configurar métodos de obtención de medidas representativas de temperatura & El sistema debe permitir configurar la forma en que se establece la representatividad de las múltiples medidas de temperatura obtenidas por los diferentes sensores (media, promedio, etc).
     \\\hline
    \end{tabularx}
    \label{ReqFuncionales_Parte3}
    \end{center}
    \end{minipage}
    
    \begin{minipage}{0.95\textwidth}
    \begin{center}
    \begin{tabularx}{\textwidth}{| X | X | X |}
    \hline
    \multicolumn{3}{|c|}{\textbf{Requerimientos no funcionales}} \\
    \hline
    ID & Nombre & Descripción \\
    \hline
    \hline
         RNF001 & Usabilidad & El sistema debe contar con una interfaz intuitiva, información útil y clara de leer. \\\hline
         RNF002 & Costo de Hardware & El componente de hardware debe ser implementado reduciendo los costos de desarrollo.
         \\\hline
         RNF003 & Elección de Plataforma para software & El SO para el que se desarrolle el componente de software debe ser el de mayor acceso en el mercado Argentino.
         \\\hline
         RNF004 & Demoras aceptables de transmisión entre componentes & El método de interfaz debe permitir la transmisión de medidas de la experiencia dentro de intervalos aceptables de demora.
         \\\hline
    \end{tabularx}
    \label{ReqNoFuncionales_Parte1}
    \end{center}
    \end{minipage}    
         
    
    \begin{minipage}{0.95\textwidth}
    \begin{center}
    \begin{tabularx}{\textwidth}{| X | X | X |}
    \hline
    \multicolumn{3}{|c|}{\textbf{Requerimientos no funcionales cont.}} \\
    \hline
    ID & Nombre & Descripción \\
    \hline
    \hline     
        RNF005 & Recuperación de datos ante desconexiones & El sistema debe ser capaz de almacenar en el componente de hardware una copia de seguridad de los datos de la experiencia que se esté realizando, de manera de no perder datos antes eventuales pérdidas de conexión.
        \\ \hline
        RNF006 & Tamaño reducido & El componente electrónico debe tener un tamaño reducido, de manera de no estorbar las actividades de producción.
         \\\hline
    \end{tabularx}
    \label{ReqNoFuncionales_Parte2}
    \end{center}
    \end{minipage}
    
    \begin{minipage}{0.95\textwidth}
    \par
    A partir de estos requerimientos, se identificó el actor del sistema y las
    interacciones que debe realizar con el mismo:
    \par
    \textbf{Cervecero}: Es el usuario del sistema, quien se encarga de definir las planificaciones de maceración, utilizar el monitoreo de variables para dar seguimiento al proceso y realizar análisis a partir de las experiencias y la información obtenida a partir de estas.
    \end{minipage}
 
 
    
    \subsection{Casos de uso}
 
    \begin{minipage}{0.95\textwidth}
    \begin{center}
    \begin{tabularx}{\textwidth}{ | X | X |}
        \hline
        \multicolumn{2}{|c|}{\textbf{Caso de Uso: Monitorear variables - CU001}} \\
        \hline
        \multicolumn{2}{|l|}{Actor: Productor de Cerveza} \\
        \hline
        Curso Normal & Curso Alternativo \\
        \hline
        1- El CU comienza cuando el usuario hace click en el botón “Monitorear Variables”. & \\
        \hline
        2- El sistema muestra una pantalla con los valores de temperatura y pH siendo medidos. & 2.1- El sistema no detecta los sensores. 
        El sistema muestra el mensaje “No pudieron obtenerse los datos del componente de Hardware. Verificar conexión “. Vuelve a la pantalla anterior.
        \\
        \hline
    \end{tabularx}
    \label{CU001}
    \end{center}
    \end{minipage}
    
    
    \begin{minipage}{0.95\textwidth}
    \begin{center}
    \begin{tabularx}{\textwidth}{ | X | X |}
        \hline
        \multicolumn{2}{|c|}{\textbf{Caso de Uso: Planificar una maceración - CU002}} \\
        \hline
        \multicolumn{2}{|l|}{Actor: Productor de Cerveza} \\
        \hline
        Curso Normal & Curso Alternativo \\
        \hline
        1- El CU comienza cuando el usuario hace click en el botón “Planificar maceración”. & \\
        \hline
        2- El sistema muestra un panel de opciones relativas a la actividad de realización de una planificación de macerado. & 
        \\
        \hline
    \end{tabularx}
    \label{CU002}
    \end{center}
    \end{minipage}
    
    
    \begin{minipage}{0.95\textwidth}
    \begin{center}
    \begin{tabularx}{\textwidth}{ | X | X |}
        \hline
        \multicolumn{2}{|c|}{\textbf{Caso de Uso: Calcular insumos - CU003}} \\
        \hline
        \multicolumn{2}{|l|}{Actor: Productor de Cerveza} \\
        \hline
        Curso Normal & Curso Alternativo \\
        \hline
        1- El CU comienza cuando el usuario hace click en el botón “Calcular insumos”. & \\
        \hline
        2- El sistema despliega un panel para que el usuario seleccione la maceración de la que obtendrá los datos. &  \\
        \hline
        3- El usuario selecciona la maceración deseada. &  \\
        \hline
        4- El sistema muestra la cantidad de malta de cada tipo a utilizar, el volumen de agua requerido en base teórica y práctica. Además despliega los respectivos rendimientos del grano(teórico y práctico). & 4.1- En caso de no haberse repetido más de 3 experimentos con la maceración seleccionada, no se mostrará el rendimiento práctico ni la cantidad de insumos en base práctica. El sistema mostrará en su lugar el mensaje “Experiencias Insuficientes”.  \\
        \hline
        
    \end{tabularx}
    \label{CU003}
    \end{center}
    \end{minipage}
    
    
    \begin{minipage}{0.95\textwidth}
    \begin{center}
    \begin{tabularx}{\textwidth}{ | X | X |}
        \hline
        \multicolumn{2}{|c|}{\textbf{Caso de Uso: Cargar nueva maceración - CU004}} \\
        \hline
        \multicolumn{2}{|l|}{Actor: Productor de Cerveza} \\
        \hline
        Curso Normal & Curso Alternativo \\
        \hline
        1- El CU comienza cuando el usuario hace click en el botón “Nueva Maceración”. & \\
        \hline
        2- El sistema despliega un formulario con los datos necesarios para la realización de una maceración. &
        \\
        \hline
        3- El usuario ingresa el nombre identificativo de la maceración, el volúmen de producción deseada, las características y porcentajes de la/s malta/s que se utilizarán, y el tipo de maceración que será realizada. Finalmente presiona el botón “Guardar”. &
        \\
        \hline
        4- El sistema emitirá un mensaje indicando que los datos ingresados han sido guardados exitosamente.  & 
        4.1- Ya existe este nombre en la base de datos. El sistema mostrará una alerta con el mensaje “Nombre ya utilizado, ingrese uno diferente”.\newline 4.2- Se utilizaron caracteres inválidos para rellenar los espacios. El sistema mostrará una alerta mencionando el dato inválido y una lista de los valores inválidos.
        \\
        \hline
    \end{tabularx}
    \label{CU004}
    \end{center}
    \end{minipage}
    
    
    \begin{minipage}{0.95\textwidth}
    \begin{center}
    \begin{tabularx}{\textwidth}{ | X | X |}
        \hline
        \multicolumn{2}{|c|}{\textbf{Caso de Uso: Comenzar maceración - CU005}} \\
        \hline
        \multicolumn{2}{|l|}{Actor: Productor de Cerveza} \\
        \hline
        Curso Normal & Curso Alternativo \\
        \hline
        1- El CU comienza cuando el usuario hace click en el botón “comenzar maceración”. & \\
        \hline
        2- El sistema muestra un menú para que el usuario indique que maceración ya cargada en el sistema va a realizar. Luego el sistema muestra los insumos requeridos para esa maceración. & 2.1 En caso de no haber realizado más de tres maceraciones para la maceración indicada, el sistema solo mostrará los cálculos basados en valores teóricos.
        \\
        \hline
        3- El usuario clickea el botón “Comenzar Maceración”, se invoca el caso de uso “Monitorear la maceración” & 
        \\
        \hline
    \end{tabularx}
    \label{CU005}
    \end{center}
    \end{minipage}
    
    
    \begin{minipage}{0.95\textwidth}
    \begin{center}
    \begin{tabularx}{\textwidth}{ | X | X |}
        \hline
        \multicolumn{2}{|c|}{\textbf{Caso de Uso: Ver datos históricos - CU006}} \\
        \hline
        \multicolumn{2}{|l|}{Actor: Productor de Cerveza} \\
        \hline
        Curso Normal & Curso Alternativo \\
        \hline
        1- El CU comienza cuando el usuario hace click en el botón “Datos Históricos”. & \\
        \hline
        2- El sistema muestra una pantalla con un menú desplegable para que el usuario seleccione el nombre de la Maceración de la que desea obtener los datos históricos. & 2.1- No hay ninguna maceración cargada. El sistema alerta la situación y bloquea las opciones de selección de maceración. Vuelve a la pantalla principal.\\
        \hline
        3- El usuario selecciona el nombre de la Maceración de la que desea obtener los datos históricos. &
        \\
        \hline
        4- El sistema muestra debajo del menú desplegable, gráficas y datos relativos a estadísticas en relación a las experiencias realizadas de esta Maceración. Al final de la pantalla se muestran dos botones, “Histórico Completo” y “Ver detalle de la Maceración”. & 4.1- No existen experiencias históricas para la maceración seleccionada. El sistema despliega una alerta con el mensaje “ No se registraron Experiencias previas para esta Maceración, por favor seleccione otra”. Bloquea la opción “Histórico completo”.\newline
        4.2- El usuario hace click en el botón “Histórico Completo”, se invoca el caso de uso “Ver Histórico Completo CU008”.\newline
        4.3-El usuario hace click en el botón “Ver detalle de la Maceración”, se invoca el caso de uso “Ver Detalle de la Maceración CU009”.
        \\
        \hline
    \end{tabularx}
    \label{CU006}
    \end{center}
    \end{minipage}
    
    
    \begin{minipage}{0.95\textwidth}
    \begin{center}
    \begin{tabularx}{\textwidth}{ | X | X |}
        \hline
        \multicolumn{2}{|c|}{\textbf{Caso de Uso: Ver Histórico Completo - CU007}} \\
        \hline
        \multicolumn{2}{|l|}{Actor: Productor de Cerveza} \\
        \hline
        Curso Normal & Curso Alternativo \\
        \hline
        1- El CU comienza cuando el usuario hace click en el botón ”Histórico Completo”. & \\
        \hline
        2- El sistema despliega un panel con gráficas de temperatura, pH y activación enzimática. En un segundo panel muestra gráficas de temperatura promedio/media, de cada experimento, ordenadas en forma descendente por fecha. Precedidas por la fecha de cada Experimento. & 
        \\
        \hline
    \end{tabularx}
    \label{CU007}
    \end{center}
    \end{minipage}
    
    
    \begin{minipage}{0.95\textwidth}
    \begin{center}
    \begin{tabularx}{\textwidth}{ | X | X |}
        \hline
        \multicolumn{2}{|c|}{\textbf{Caso de Uso: Ver detalle de maceración - CU008}} \\
        \hline
        \multicolumn{2}{|l|}{Actor: Productor de Cerveza} \\
        \hline
        Curso Normal & Curso Alternativo \\
        \hline
        1- El CU comienza cuando el usuario hace click en el botón ”Ver Detalle de la Maceración”. & \\
        \hline
        2- El sistema muestra una pantalla con los datos ingresados cuando fue creada la maceración seleccionada (CU005 “Cargar Nueva Maceración”). & 
        \\
        \hline
    \end{tabularx}
    \label{CU008}
    \end{center}
    \end{minipage}
    
    
    \begin{minipage}{0.95\textwidth}
    \begin{center}
    \begin{tabularx}{\textwidth}{ | X | X |}
        \hline
        \multicolumn{2}{|c|}{\textbf{Caso de Uso: Monitorear experimento de maceración - CU009\_Parte1}} \\
        \hline
        \multicolumn{2}{|l|}{Actor: Productor de Cerveza} \\
        \hline
        Curso Normal & Curso Alternativo \\
        \hline
        1- El CU comienza cuando el usuario elige la opción “Comenzar experimento”. & \\
        \hline
        2- El sistema despliega un panel donde el usuario podrá visualizar las variables involucradas durante el proceso. El sistema queda en reposo hasta que el usuario indique que da inicio al monitoreo de la maceración. Además, contará con un campo  final del experimento de maceración que se llevará a cabo. &
        \\
        \hline
        3- El usuario elige la opción “comenzar monitoreo”. & 3.1- Hubo un problema con la conexión de los sensores. El sistema alerta al usuario de la situación y no comienza la monitorización del proceso.
        \\
        \hline
        \end{tabularx}
    \label{CU009_a}
    \end{center}
    \end{minipage}
    
    \begin{minipage}{0.95\textwidth}
    \begin{center}
    \begin{tabularx}{\textwidth}{ | X | X |}
    \hline
        \multicolumn{2}{|c|}{\textbf{Caso de Uso: Monitorear experimento de maceración - CU009\_Parte2}} \\
        \hline
        \multicolumn{2}{|l|}{Actor: Productor de Cerveza} \\
        \hline
    \hline
        Curso Normal & Curso Alternativo \\
        \hline
        4- El sistema muestra los valores de temperatura actual y planificada para el momento que transcurre, el tiempo que ha transcurrido y el faltante para el próximo cambio de temperatura planificado, pH, Actividad enzimática. & 4.1- Una vez pasado el tiempo completo planificado para la maceración elegida, el sistema muestra la opción finalizar maceración.
        \\
        \hline
        5- El usuario elige la opción “Detener monitoreo”. &
        \\
        \hline
        6- El sistema se mantiene en reposo para que el usuario pueda cargar la densidad final obtenida. &
        \\
        \hline
        7- El usuario elige la opción finalizar la maceración. &
        \\
        \hline
        8 - El sistema indica que los datos de la maceración han sido almacenados correctamente. &
        \\
        \hline
    \end{tabularx}
    \label{CU009_b}
    \end{center}
    \end{minipage}
    
    
    \begin{minipage}{0.95\textwidth}
    \begin{center}
    \begin{tabularx}{\textwidth}{ | X | X |}
        \hline
        \multicolumn{2}{|c|}{\textbf{Caso de Uso: Borrar datos de maceración - CU010}} \\
        \hline
        \multicolumn{2}{|l|}{Actor: Productor de Cerveza} \\
        \hline
        Curso Normal & Curso Alternativo \\
        \hline
        1- El CU comienza cuando el usuario hace click en el botón “Borrar datos de esta maceración” en el panel de Maceraciones. & \\
        \hline
        2- El sistema despliega un panel con un mensaje que indique que se perderán todos los datos relacionados a esa maceración y opciones para aceptar o cancelar. &
        \\
        \hline
        3- El usuario elige una de las opciones del panel. & 3.1- El usuario elige la opción “Aceptar”. El sistema elimina todos los datos y emite una notificación que los datos han sido eliminados correctamente.\newline
        3.2- El usuario elige la opción “Cancelar”. El sistema vuelva a “Ver Maceraciones”.
        \\
        \hline
    \end{tabularx}
    \label{CU010}
    \end{center}
    \end{minipage}
    
    
    \begin{minipage}{0.95\textwidth}
    \begin{center}
    \begin{tabularx}{\textwidth}{ | X | X |}
        \hline
        \multicolumn{2}{|c|}{\textbf{Caso de Uso: Borrar experimento de maceración - CU011}} \\
        \hline
        \multicolumn{2}{|l|}{Actor: Productor de Cerveza} \\
        \hline
        Curso Normal & Curso Alternativo \\
        \hline
        1- El CU comienza cuando el usuario hace click en el botón "Borrar Experimentos" perteneciente al panel del Histórico Completo. & \\
        \hline
        2- El sistema despliega una alerta con el mensaje “Está seguro que desea eliminar estos experimentos? (Los mismos serán eliminados de forma permanente)”. &
        \\
        \hline
        3- El usuario presiona el botón “Confirmar borrado”. & 3.1- El usuario hace click en el botón “Cancelar”.
        \\
        \hline
        4- El sistema elimina los experimentos y regresa al caso de uso “Ver Histórico Completo - CU008”. & 4.1- El sistema vuelve al caso de uso “Ver Histórico Completo - CU008”.
        \\
        \hline
    \end{tabularx}
    \label{CU011}
    \end{center}
    \end{minipage}
    
    \subsubsection{Diagrama de Casos de uso}
    El mismo se encuentra presente en el Anexo, Figura \ref{DiagCU}.

	
    
    
    
    

    