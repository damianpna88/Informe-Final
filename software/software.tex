\chapter{Software}
\par El componente de software tendrá el propósito de asistir al fabricante de cerveza en las tareas de monitoreo, planificación y visualización de datos históricos de experimentos de maceración.

\par En este capítulo se describen las consideraciones tomadas para la elección de la plataforma sobre la cual se desarrolló la aplicación, el diseño conceptuado y, finalmente, una descripción detallada del desarrollo realizado.

\section{Análisis de alternativas}
    En esta sección se argumenta la elección de la plataforma sobre la cual es desarrollado este componente.
    
    \subsection{Tecnologías}
        \par A continuación se presenta una breve reseña de las plataformas actualmente más relevantes para dispositivos móviles.
        
        \subsubsection{Android}
            \par Android es un sistema operativo, el cual fue inicialmente desarrollado por Android Inc.\footnote{Sitio oficial: \url{https://www.android.com/}}, empresa que Google\textsuperscript{\textregistered} respaldó económicamente y más tarde, en 2005, compró. Fue diseñado principalmente para dispositivos móviles con pantalla táctil desarrollados por Google\textsuperscript{\textregistered} o por terceros, como teléfonos inteligentes, tabletas y también, relojes inteligentes, televisores y automóviles. Android es un sistema operativo basado en el núcleo Linux.
            
        \subsubsection{iOS}
            \par iOS es un sistema operativo móvil de la multinacional Apple Inc\footnote{Sitio oficial: \url{https://www.apple.com/}}. Originalmente desarrollado para el teléfono móvil \textit{iPhone} (iPhone OS), después utilizado en dispositivos como el \textit{iPod touch} y el \textit{iPad}. Este sistema operativo móvil solo puede ser instalado sobre dispositivos móviles pertenecientes a Apple Inc. iOS se deriva de macOS\footnote{Sistema operativo de Apple para computadoras portátiles y de escritorio \url{https://www.apple.com/la/macos/mojave/}}, que a su vez está basado en Darwin BSD, y por lo tanto es un sistema operativo Tipo Unix.
            
    \subsection{Comparación}
        \par Las consideraciones a tener en cuenta para la elección de la plataforma móvil para el desarrollo son las siguientes: cantidad de usuarios que la utilicen; tamaño de la comunidad a disposición para dar soporte; disponibilidad de mejores herramientas y utilidades para el desarrollo de aplicaciones.
        
        \par En la Tabla \ref{tab:ComparacionPlataformasMoviles} se presenta un cuadro comparativo entre ambas tecnologías dónde se abordan aspectos de interés para la elección y un análisis concluyente.
        
        \begin{table}[h]
            \centering
            \begin{tabularx}{\textwidth}{|X|X|X|}
                 \hline
                 \multicolumn{3}{|c|}{Tabla comparativa de tecnologías de software}\\
                 \hline
                 Criterios de comparación & Android & iOS \\
                 \hline
                 \hline
                 
                 Porcentaje Mercado (Arg) & 75\% & 19\%  \\
                 \hline
                 
                 Porcentaje Mercado Internacional & 85\% & 14,7\% \\
                 \hline
                 
                 Comunidad de desarrolladores & Muy grande & Amplia pero acorde al número de usuarios\\
                 \hline
                 
                  Entornos desarrollo Propia & Android Studio & Xcode\\
                 \hline
                 
                 Lenguaje de desarrollo & Java, C, C++ y Kotlin & Swift, C, C++ y objective-C\\
                 \hline
                 
                 Familia del SO & Linux & Unix - BSD\\
                 \hline
                 
                 Complejidad de desarrollo & Alta variedad de dispositivos y versiones del SO & Baja variedad de dispositivos, versiones de SO comunes a la mayoría\\
                 \hline
                 
                 Entorno & Open Source & Entorno cerrado \\
                 \hline
                 
                 Requerimientos para publicación de aplicación & Ninguna & Debe cumplir requisitos de Apple\\
                 \hline
                 
            \end{tabularx}
            \caption{Comparación de plataformas móviles}
            \label{tab:ComparacionPlataformasMoviles}
        \end{table}

    
    \subsection{Elección}
    \par
    A partir del cuadro comparativo de la Tabla \ref{tab:ComparacionPlataformasMoviles} y los aspectos aquí mencionados se decide optar por la plataforma Android. 
    %La elección fue realizada a partir de los criterios antes mencionados aplicados sobre la tabla comparativa.
    \par
    %Se decidió desarrollar la aplicación para la plataforma \textbf{Android}, considerado, 
    Esta decisión se basa en los siguientes aspectos: El alto porcentaje de uso en Argentina, el tamaño de la comunidad de desarrolladores, la gran disponibilidad de herramientas y la sencillez relativa en cuanto a requerimientos para la publicación de aplicaciones.
    
    \par Para el desarrollo se emplea el lenguaje Java (JDK 1.7 o Java 7) sobre el entorno de desarrollo integrado (IDE) oficial de Google, AndroidStudio. La versión mínima de sistema operativo para la que se desarrolla es \textit{Nougat 7.0}, lanzada en agosto de 2016, la cual utiliza la plataforma de desarrollo (Android SDK plataform \footnote{\url{https://developer.android.com/studio/releases/platforms}}) API nivel 24.

    
    
\section{Diseño}
    
    \par En esta sección es presentan los modelos diseñados para la interfaz gráfica y la base de datos del dispositivo móvil.
    
    \subsection{Diseño de la interfaz de usuario}
        \par Las figuras \ref{fig:MockUpMainActivity}, \ref{fig:MockUpPlanningActivity}, \ref{fig:MockUpExperimentActivity}, \ref{fig:MockUpInfoMash}, \ref{fig:MockUpCurrentExperienceFragment}, \ref{fig:MockUpStageFragment}, \ref{fig:MockUpGeneralFragment}, \ref{fig:MockUpExperimentFragment} y \ref{fig:MockUpDetailExperimentActivity} ubicadas en el Anexo, modelan y describen el diseño y las funcionalidades de la interfaz de usuario (UI) bosquejadas para la aplicación móvil.

    \subsection{Base de datos}
        \par En el diagrama presente en la Figura \ref{fig:DiagramaBdApp} se encuentran las tablas diseñadas para la base de datos y el conjunto de relaciones establecidas entre ellas.
        
        \begin{figure}[h]
            \centering
            \includegraphics[scale=0.8]{DiagramaBaseDeDatosAPP.jpg}
            \caption{Modelo de Base de Datos de la aplicación}
            \label{fig:DiagramaBdApp}
        \end{figure}

\section{Implementación}
    \par En esta sección se realiza una breve introducción al desarrollo Android, de manera de construir una base de conocimientos del tema. Posteriormente se realiza una descripción estética y funcional de cada pantalla de la interfaz gráfica creada y un detalle de su implementación. 
    %Luego, para cada pantalla implementada se realizará una descripción estética y funcional junto a un detalle de su implementación.
    
    \subsection{Breve introducción al desarrollo en Android}
    
    \par Las aplicaciones en Androdid son desarrolladas siguiendo el patrón de arquitectura de desarrollo MVP (Modelo-Vista-Presentador). Este patrón separa las lógicas de datos y negocios (modelo y presentador respectivamente) de la interfaz gráfica (Vista). En este contexto, la lógica es llevada a cabo por elementos denominados \textit{Activities} y Adaptadores (\textit{Adapters}), y la interfaz gráfica por componentes denominados Vistas(\textit{Views}). En cada clase \textit{Activity} se define la lógica de funcionamiento de la aplicación y el manejo de una pantalla relacionada a ésta con la que el usuario interactúa. Entre las funcionalidades implementadas dentro de una \textit{Activity} pueden mencionarse principalmente: gestión de widgets\footnote{Elementos gráficos preestablecidos que implementan funcionalidades, como menús desplegables, botones, relojes, etc.} desplegados en la pantalla; lógica o dinámica de funcionamiento; manejo de información ingresada y desplegada en pantalla; llamadas a librerías externas de interacción con bases de datos, APIs o incluso otros \textit{Activities}.
    
    % ------ Ver si es necesario hablar de ciclo de vida del activity.
    \subsubsection{Interfaz gráfica}
    \label{explicacionInterfazGrafica}
    
    \par Una estructura de pantalla básica está compuesta por un \textit{Actionbar} y un cuerpo (Véase Figura \ref{fig:emptyActivity}). El \textit{Actionar} se ubica en la parte superior y contiene el título de la pantalla o aplicación y soporta el agregado de un \textit{OptionsMenu} (compuesto por botones y/o un menú desplegable). En el cuerpo se ubica el contenedor de Vistas (\textit{layout}) y el contenido gráfico-funcional definido para la pantalla (\textit{widgets}).
    
    \begin{figure}
        \centering
        \includegraphics[scale=0.5]{software/EmptyActivity.jpg}
        \caption{\textit{EmptyActivity} - Pantalla básica en Android: ActionBar(Azul con texto \textit{EmptyActivityDemo}) y Cuerpo (Blanco con texto \textit{Hello World!})}
        \label{fig:emptyActivity}
    \end{figure}
    
    %\begin{minipage}[0.95\textwidth]
    \par Mediante el uso de layouts, definidos en archivos XML, se estructura la pantalla a ser presentada (posiciones, margenes, etc.). Existen distintos tipos de layouts que incorporan con su uso distintas funcionalidades. El comportamiento final de las pantallas, se define entonces, a partir del anidamiento de distintos layouts de manera de poder utilizar las ventajas que cada tipo implementa. Dentro de los layouts disponibles se encuentran:
    
    \begin{itemize}
        \item \textit{LinearLayout}, define en secuencia vertical (arriba hacia abajo)  u horizontal (izquierda a derecha) el orden en que se visualizan los componentes;
        
        \item \textit{RelativeLayout}, define la ubicación en pantalla del componente en referencia a otro componente ya añadido;
        
        \item \textit{FrameLayout}, define la ubicación de los componentes de manera absoluta, permitiendo la superposición de los mismos;
        
        \item \textit{ScrollView}, utilizado cuando el espacio en pantalla es menor que el contenido que se desea mostrar, ofreciendo la posibilidad de desplazamiento para visualizar el área de interés del contenido;
        
        \item \textit{ListView}, genera una lista de \textit{layout} anidados, utilizado cuando la cantidad de estos últimos varia dinámicamente. Para el manejo de los mismos, se utiliza un tipo de clase denominado \textit{Adapter} que gestiona la información de cada entrada de la lista y como se visualiza dentro de la misma;
        
        \item \textit{RecyclerView}, posee la misma funcionalidad que el \textit{ListView} pero este no renderiza los layout que no se muestran en pantalla, a diferencia del ListView. Se utiliza cuando la cantidad de \textit{layout} anidados es mayor a la que puede ser mostrada en pantalla. Este \textit{layout} evita que el requerimiento de memoria no se vea afectado por la cantidad de entradas y se reduzca el rendimiento de la aplicación;
        
        \item \textit{CardView}, es un contenedor estético que usualmente presenta un panel de bordes redondeados y sombra, muy utilizado en conjunto con \textit{RecyclerView} o \textit{ListView} como layout anidado dentro los mismos.
        
    \end{itemize}

    %\end{minipage}
    
    \par Los componentes de tipo widget, antes mencionados, son definidos dentro de los layouts. Mediante el uso de cada uno se especifica una estructura en particular, entre estos fueron utilizados: botones estáticos (\textit{Button}) o flotantes (\textit{FloatingActionButton}), menús desplegables (\textit{Spinner}), campos para inserción y presentación de texto (\textit{EditText y TextView} respectivamente), cronómetros (\textit{Chronometer}).
        
    \par Una pantalla puede incorporar, además de \textit{Views}, cuadros de diálogo (pop-up) emergentes y menús contextuales (\textit{ContextMenu}) que son invocados a través de la interacción del usuario con un \textit{View}. Para la definición del contenido visual y el comportamiento de estos diálogos se utiliza la clase \textit{AlertDialog} para una ventana tipo emergente y métodos propios del \textit{Activity} para un menú contextual. Ambos casos se implementan dentro del \textit{Activity}, que es el encargado de gestionar esa pantalla.
        
    \par Como fue antes mencionado, la lógica funcional de una pantalla es implementada en un \textit{Activity}. Dentro de este pueden ser incluidos \textit{Fragments}, los cuales representan un comportamiento o una parte de la interfaz de usuario. Mediante el uso combinado de Activity y Fragments en una misma actividad es posible crear interfaces gráficas multipanel. El manejo de las pantallas o interfaces definidas para un \textit{Fragment} se realiza de la misma forma que las pantallas definidas para un \textit{Activity}. Una forma de utilizar \textit{Fragments} es mediante pestañas (\textit{Tabs}). De esta manera una misma pantalla contiene un conjunto definido de cuerpos intercambiables (subpantallas). Para poder ser utilizadas por el usuario se utilizan \textit{TabLayouts}, que definen el conjunto de pestañas en la pantalla junto a clases instanciadas por el \textit{Activity} padre. El \textit{ViewPagerAdapter} es el encargado de gestionar el intercambio entre los \textit{TabLayouts} y los \textit{Activities} que las instancian.
    %encargados de gestionar los intercambios entre las mismas (\textit{ViewPagerAdapter}).
    
    \subsubsection{Librerías utilizadas}
     \par Los \textit{Activities}, dentro de su hilo de ejecución, incluyen métodos o llamadas que interaccionan con librerías o clases complementarias. A continuación se nombran las más destacadas o importantes: %Se considera importante mencionar las siguientes:
     \begin{itemize}
         %% Ver con dami
         \item \textbf{Clases POJO} (Plain Old Java Object): Son clases simples que permiten modelar las reglas de negocio a través de la estructuración y relación de la información presentes en ellas.
         %% Ver con dami
         %Para el modelado de las reglas de negocio se utilizan clases complementarias que estructuran y relacionan la información requerida mediante la definición de sus atributos y métodos.
         
         \item \textbf{Persistencia}: Es el conjunto de datos que se almacena y se mantiene aún cuando no este en ejecución la aplicación. Para el desarrollo de aplicaciones Android pueden ser utilizadas distintas librerías\footnote{Página de referencia \url{https://developer.android.com/guide/topics/data/data-storage}}. En esta aplicación se persisten datos y preferencias compartidas (datos de configuración). Los datos se gestionan mediante base de datos SQL en Android utilizando \textit{SQLite}\footnote{SQLite, sistema de gestión de bases de datos relacional}. En su implementación, se define una clase con patrón \textit{Singleton}\footnote{Singleton, patrón de diseño que permite restringir la creación de objetos pertenecientes a una clase o el valor de un tipo a un único objeto} que define la estructura de tablas de la base de datos y en la que, adicionalmente, se pueden implementar funciones para actualizar, insertar, eliminar y consultar datos de las tablas. Las preferencias compartidas se persisten mediante la utilización de \textit{SharedPreferences} donde se define un sistema clave-valor (\textit{key-value}) para persistir datos aislados cuando la aplicación se cierra.
         
         %% Dami, te regalo la definión de REST API
         \item \textbf{REST API}: Con el fin de interactuar con la API REST presentada en el Capítulo \ref{capituloInterfaz} existen múltiples opciones a ser utilizadas. En este proyecto, la interacción con la REST API, se realiza mediante la librería \textit{Retrofit} \footnote{Librería implementada por \textit{Square}. Repositorio \url{https://square.github.io/retrofit/}}, con la cual se definen los protocolos o servicios de comunicación HTTP, GET o POST. El envío y recepción de datos se realiza mediante objetos Json (\textit{JsonObject y JsonArray}), y una interfaz de conexión en la que se definen estas operaciones y la IP del servidor. Los objetos enviados por la API REST y recibidos por la clase \textit{Retrofit} deben ser transformados en clases capaces de ser comprendidas por la aplicación. Estos últimos se denominan clases contenedoras (\textit{Containers}) y se utilizan para  definir los métodos de obtención de datos.
        
        %% esto de los hilos de ejecucoón parece que hay que darle una reestructuración completa. 
        \item \textbf{Hilos de ejecución independientes}: Los hilos de ejecución independientes (Threads) permiten liberar de carga al hilo principal cuando éste se ve sobrecargado o detiene su ejecución por espera de respuestas. 
        
        \par En esta aplicación son utilizados dos tipos de hilos, hilos paralelos (\textit{Threads}) y subhilos (\textit{subThreads}). Los primeros son utilizados cuando se requieren tareas a ser ejecutadas en forma paralela al hilo de ejecución principal de la Aplicación. Estas tareas carecen de capacidad de interacción con las interfaces visuales, pero poseen la ventaja de continuar su ejecución aun si el proceso que las inició se encuentra inactivo o en segundo plano. Para el manejo de hilos paralelos se utilizó la librería \textit{WorkManager}\footnote{Página de referencia \url{https://developer.android.com/topic/libraries/architecture/workmanager/}}. En esta librería se define la funcionalidad que se va a ejecutar y el intervalo de tiempo de vida que posee. Por su parte, los subhilos no cuentan con la habilidad de ejecución independiente sino que son ejecutados junto con el hilo principal, pudiendo de esta manera actualizar la interfaz gráfica mientras en el hilo principal se llevan adelante tareas de gestión de la interfaz. Los subhilos son definidos y utilizados mediante herramientas propias del sistema Android, y no de una librería externa.
        
        \item \textbf{Gráficas estadísticas}: Para la presentación de gráficas se utiliza la librería \textit{MPAndroidChart}\footnote{Repositorio \url{https://github.com/PhilJay/MPAndroidChart}}, que dispone de todos las gráficas necesarias para esta aplicación. Para la utilización de las gráficas se dispone de un componente \textit{View} que se ubica en un de tipo \textit{ChartView} en el \textit{layout} correspondiente, el cual varía según el tipo de gráfica que se desee utilizar (Ej. \textit{LineChartView} para gráficos de líneas, \textit{CombinedChartView} para combinar distintos tipos de gráficos dentro de un mismo contexto gráfico, \textit{CandleStick} para gráficos tipo candelero, etc.).
        
        \item \textbf{Cálculos Varios}: Para realizar los cálculos auxiliares de rendimiento, cantidad de insumos, activación de enzimas, valores estadísticos, entre otros se utiliza una clase propia denominada \textit{Cálculos}. La misma actúa como librería, pudiendo ser invocada por numerosas clases dentro de la aplicación.
        
     \end{itemize}
     
    \subsection{Pantalla principal}
        \label{DescripPantallaPrincipal}
        \par Es la pantalla con la que inicia la aplicación. En la misma se puede visualizar la lista de maceraciones planificadas.
           
            \subsubsection{Descripción}
                
                \par Para la implementación de la pantalla principal se utiliza un \textit{Activity} denominado \textit{MainActivity}, (Diagrama \ref{fig:DiagClaseMainActivity}). La interfaz definida se compone por un \textit{ActionBar} que cuenta con un botón para acceder a la pantalla de planificación, una lista de elementos (maceraciones) que ocupa el cuerpo de la pantalla y un botón flotante para acceder un dialogo tipo \textit{pop-up} donde puede observarse los valores actuales recolectados por los sensores de la estación de recolección de datos (Figura \ref{fig:CapturaMainAct}). 
                
                \begin{figure}[h]
                    \centering
                    \includegraphics[scale=0.2]{software/ScreenCapture/MainActivity.jpg}
                    \caption{Captura de la pantalla principal}
                    \label{fig:CapturaMainAct}
                \end{figure}
                \begin{figure}[h]
                    \centering
                    \includegraphics[scale=0.2]{software/ScreenCapture/ShowCurrentValues.jpg}
                    \caption{Captura de la pantalla principal con el diálogo de valores actuales}
                    \label{fig:CapturaShowCurrentValues}
                \end{figure}
                
                
            \subsubsection{Funciones}
                \begin{itemize}
                    \item \textbf{Lista de maceraciones planificadas:} Cada una de las maceraciones listadas es un botón clickeable que permite el acceso a la pantalla de gestión de la maceración seleccionada.
                    
                    \item \textbf{Agregar nueva maceración:} Al seleccionar el botón destinado a este fin, se accede a la pantalla que permite crear una nueva maceración.
                    
                    \item \textbf{Informar valores actuales:} Los valores de medición obtenidos por los sensores de la estación de recolección se muestras mediante un dialogo por-up. (Figura \ref{fig:CapturaShowCurrentValues}).

                \end{itemize}
                
            \subsubsection{Detalle de implementación}
                % captura de pantalla del mainactivity o citamos el mockup?
                La lista de maceraciones planificadas se obtiene mediante la instanciación de una clase \textit{SQLiteDatabase}. Esta clase realiza una consulta SELECT para obtener los campos \textit{nombre} y \textit{tipo} de todos los índices de la tabla \textit{Maceracion} (Figura \ref{fig:DiagramaBdApp}). Una vez obtenida esta lista, se instancia una clase \textit{Adapter}\footnote{Clase que gestiona la información de cada entrada de la lista y como se visualiza dentro de la misma} que se incorpora a un objeto \textit{RecylcerView} para que inserte los datos en objetos \textit{CardViews} que son anidados dentro de la lista. Cada componente \textit{CardView} tiene la funcionalidad de iniciar la pantalla de gestión de maceración al ser presionado (subsección \ref{DescripPantallaPlanificación}).
                
                \par La información de los sensores que se muestra en el panel de Valores Actuales se obtiene a partir una llamada a la API "GetTempPh.php" (función \textit{getTempPh()} incluida en la interfaz API) mediante el uso de la librería \textit{Retrofit}. Una vez obtenidos estos valores, son insertados en una instancia de la clase \textit{TempPh} (\textit{Container}) y se carga un objeto \textit{AlertDialog} con los mismos (Figura \ref{fig:CapturaShowCurrentValues}).
                
                \par El acceso a la pantalla de planificación se realiza mediante una funcionalidad implementada para el icono del \textit{OptionsMenu} que inicia el \textit{Activity PlanningActivity} (subsección \ref{DescripPantallaPlanificación}).
                
                \par En el diagrama \ref{fig:DiagClaseMainActivity} ubicado en el Anexo, pueden observarse las clases y funciones utilizadas para el despliegue de esta pantalla.
                
                
    %---------------FIN Descripción MainActivity ---------------                
                
        \subsection{Pantalla de planificación}
        \label{DescripPantallaPlanificación}
        \subsubsection{Descripción}
                \par Es la pantalla donde se planifica una nueva maceración. En la misma se pueden visualizar opciones para rellenar los campos necesarios para definir una maceración.
                
                \par Para la implementación se utiliza un \textit{Activity} denominado \textit{PlanningActivity}. Como puede observarse en la figura \ref{fig:CapturaPlanAct} la interfaz definida se encuentra compuesta por un \textit{Actionbar} que cuenta con un botón para finalizar la planificación, una lista de campos (tipo de maceración, volumen, densidad deseada, lista de granos, lista de intervalos) que ocupa el área restante de la pantalla.
                \begin{figure}[h]
                    \centering
                    \includegraphics[scale=0.2]{software/ScreenCapture/PlanningActivity.jpg}
                    \caption{Captura de la pantalla de planificación de maceración}
                    \label{fig:CapturaPlanAct}
                \end{figure}
                \begin{figure}[h]
                    \centering
                    \includegraphics[scale=0.2]{software/ScreenCapture/PlanningActivity-AddGrain.jpg}
                    \caption{Captura del dialogo para añadir un Grano en la pantalla de planificación de maceración}
                    \label{fig:CapturaPlanAddGrain}
                \end{figure}
                \begin{figure}[h]
                    \centering
                    \includegraphics[scale=0.2]{software/ScreenCapture/PlanningActivity-AddInterval.jpg}
                    \caption{Captura del dialogo para añadir un Intervalo en la pantalla de planificación de maceración}
                    \label{fig:CapturaPlanAddInterval}
                \end{figure}
                \begin{figure}[h]
                    \centering
                     \includegraphics[scale=0.2]{software/ScreenCapture/PlanningActivity-Finish.jpg}
                    \caption{Captura del dialogo para finalizar la pantalla de planificación de maceración}
                    \label{fig:CapturaPlanFinish}
                \end{figure}

               
            \subsubsection{Funciones}
                \begin{itemize}
                    \item \textbf{Seleccionar tipo de maceración:} Menú desplegable que permite seleccionar el tipo de maceración de una lista predefinida, siendo estas opciones \textit{simple}, \textit{escalonada} o \textit{decocción}.
                    
                    \item \textbf{Establecer volumen y densidad deseada:}Estos parámetros se pueden establecer rellenando los campos correspondientes. El volumen ingresado debe estar expresado en litros y la densidad en kilogramos sobre litro.
                    
                    \item \textbf{Agregar/Quitar granos:} Al presionar el botón ``AGREGAR GRANO'', se despliega un dialogo con tres campos, (Figura \ref{fig:CapturaPlanAddGrain}). Una vez agregado, los datos del mismo se mostrarán en una lista en conjunto con los granos agregados anteriormente.
                    
                        \par Para eliminar un elemento de esta lista, se debe mantener presionado el mismo y seleccionar la opción ``Eliminar grano'' del menú contextual.
                    
                    \item \textbf{Definir intervalos con sus parámetros:}Para el agregado de intervalos debe seleccionarse el botón con el signo ``+'', esto desplegará un dialogo tipo \textit{pop-up}, Figura \ref{fig:CapturaPlanAddInterval}, donde podrán encontrarse los campos necesarios para definir un intervalo. Pueden ser agregados uno o mas intervalos según corresponda al tipo y receta de maceración a ser realizada.
                        \par Para eliminar un elemento de esta lista, se debe mantener presionado el mismo por un lapso de tiempo prolongado.
                    
                    \item \textbf{Finalizar planificación:} Al finalizar el ingreso de datos, ser presionará el botón para finalizar, antes mencionado, esto desplegará un dialogo tipo \textit{pop-up} donde deberán ser ingresados los campos nombre de la maceración, intervalo de medición de temperatura e intervalo de medición de pH (Figura \ref{fig:CapturaPlanFinish}). Luego del ingreso de los datos, presionando el botón ``Aceptar'' dará por concluido el ingreso de datos y se retornará al \textit{Main Activity}.
                    
                    \item \textbf{Regla de negocios} Dado que es necesario mantener igualdad de condiciones para los experimentos de maceración llevados a cabo con los parámetros aquí ingresados, los mismos no podrán ser modificados luego. Debiendo en el caso de requerir modificarlos, eliminar la maceración e ingresar nuevamente los datos.
                \end{itemize}
            
            \subsubsection{Detalle de implementación}
                \par De manera de permitir la selección uno de los tres tipos de maceración, para esto se utilizó un \textit{Spinner} con las tres opciones: \textit{Simple}, \textit{Escalonada}, \textit{Decocción}. Luego, para indicar los valores volumen de mosto y densidad planificados, se utilizaron dos \textit{EditText}. %Todos estos con etiquetas para indicar el campo a rellenar (TextView's)
                
                \par Los granos correspondientes a la receta se agregan mediante un botón con la etiqueta ``AGREGAR GRANO'', el cual al ser presionado muestra un \textit{AlertDialog}. Este último posee tres campos (\textit{EditText}) para ser rellenados: Nombre de grano, Porcentaje de utilización y Extracto Potencial. Cuando se presiona el botón ACEPTAR de este \textit{pop-up}, se realiza una validación de campos ingresados (negando la aceptación en caso de incompletitud) con el fin de agregar una nueva entrada en un \textit{ListView} dentro del layout del \textit{PlanningActivity}. Cada entrada en esta lista posee un \textit{TextView} cargado con la descripción del grano. Se permite la eliminación de cualquier entrada, mediante el procedimiento de interacción con un \textit{ContextMenu} ya descripto en el apartado de funciones.
                
                \par Los intervalos de la maceración también son agregados mediante un \textit{AlertDialog}, que es invocado al seleccionar el \textit{FloatingActionButton} presente en el layout del \textit{Activity}. Este último contiene un conjunto de \textit{EditText} para completar los datos del intervalo: duración; temperatura; desvío tolerado de temperatura; pH; desvío tolerado de pH. En caso que el tipo de maceración seleccionada sea ``Decocción'', también se incluirán los campos temperatura secundaria y desvío tolerado para temperatura secundaria. Luego de presionarse el botón ACEPTAR, se cierra la ventana emergente, y un nuevo \textit{CardView} es insertado dentro de una lista \textit{RecyclerView} donde cada entrada posee la información del intervalo agregado. Es posible eliminar cualquier entrada mediante el procedimiento ya descripto en el apartado de funcionalidades.
                
                \par En el diagrama \ref{fig:DiagClasePlanningActivity} ubicado en el Anexo, pueden observarse las clases y funciones utilizadas para el despliegue de esta pantalla.
        %---------------FIN Descripción PlanningActivity ---------------
        
        \subsection{Pantalla de gestión de maceración}
        \label{DescripPantallaGestiónMaceración}
            \subsubsection{Descripción}
                \par Es la pantalla de administración para la maceración seleccionada.
                \par Para la implementación se utiliza un \textit{Activity} denominado \textit{ExperimentActivity}. En la figura \ref{fig:CapturaExperimentAct} puede ser visualizado en primer lugar un \textit{ActionBar} donde se indica el nombre de la maceración y que incluye además una serie de botones: acceso a la pantalla de detalle de la maceración, ingreso a la pantalla de información histórica y estadística de los experimentos realizados, y por último, la opción de eliminar la maceración. En el área restante de la pantalla puede encontrarse una lista con los experimentos realizados, y un botón flotante en la parte inferior que permite iniciar un nuevo experimento y acceder de esta manera a la pantalla correspondiente.
                \begin{figure}[h]
                    \centering
                    \includegraphics[scale=0.2]{software/ScreenCapture/ExperimentActivity.jpg}
                    \caption{Captura de la pantalla de Gestión de la Maceración}
                    \label{fig:CapturaExperimentAct}
                \end{figure}
            \subsubsection{Funciones}
                \begin{itemize}
                    \item \textbf{Lista de experimentos realizados:} Cada una de los experimentos listados es un botón sensible a ser presionado que permite el acceso a la pantalla de resultados del experimento pulsado (Figura \ref{fig:CapturaDetExpAct}).
                    
                    \item \textbf{Eliminar experimento:} Como se mencionó en el punto anterior, cada una de los experimentos listados es un botón sensible a ser presionado. Si la duración de esta selección supera los dos segundos, se procede a borrar el experimento seleccionado de la lista y de la base de datos.
                    
                    \item \textbf{Detalle del plan de maceración:} Pantalla en la cual pueden ser visualizados los valores ingresados en la planificación de la maceración; el volumen y temperatura de agua a ser incorporada en cada etapa; y la cantidad de cada tipo de grano a ser utilizado.
                    
                    \item \textbf{Información histórica y estadística:} Acceso a la pantalla de información histórica y estadística correspondiente a la maceración. (Figura \ref{fig:CapturaGeneralFrag-P1}).
                    
                    \item \textbf{Eliminar maceración:} Al seleccionarse la opción presente en el \textit{ActionBar}, se procede a eliminar la maceración actual en conjunto con todos los datos ligados a la misma (experimentos realizados y datos inherentes a la planificación de la misma)
                    
                    \item \textbf{Iniciar nuevo experimento:} Ingreso a la pantalla de monitoreo del experimento a ser iniciado (Figura \ref{fig:CapturaMeasureFrag}).
                    
                \end{itemize}
            \subsubsection{Detalle de implementación}
                \par Se da inicio a un nuevo experimento y su correspondiente monitoreo, adjuntando el ID de maceración actual, luego de presionado el \textit{FloatingActionButton} presente en el \textit{ExperimentActivity}. Este monitoreo se describe en subsección \ref{DescripPantallaMonitoreoExperimento}.
            
                \par En la pantalla se puede observar la lista de experimentos realizados para dicha maceración. Para cargarla con los valores correspondientes, se instancia una clase \textit{SQLiteDatabase} y se realiza una consulta \textit{SELECT} para obtener los campos \textit{id} y \textit{fecha} de todos los índices de la tabla \textit{Experimento} que posean el ID de Maceración obtenido del \textit{MainActivity}. Una vez obtenida esta lista, se instancia un \textit{Adapter} y se lo adjunta a un \textit{RecylcerView}, para que inserte los datos en \textit{CardViews} que son anidados en la lista. Estos últimos tienen añadido dos funcionalidades: la primera es invocada cuando se los presiona, iniciando la pantalla de resultados del experimento (subsección \ref{DescripPantallaResultadosExperimento}) adjuntando el ID perteneciente al experimento del \textit{CardView} seleccionado. La segunda elimina el experimento, del \textit{RecyclerView} y de la base de datos, cuando el usuario mantiene presionado el ítem durante un lapso prolongado.
                
                \par Dentro del \textit{OptionsMenu} se encuentran tres opciones: \textbf{Información histórica}, mediante la cual se accede a la Pantalla de información histórica y estadística (subsección \ref{DescripPantallaEstadística}); \textbf{Detalle del plan de maceración}, que inicia la Pantalla de detalle de maceración (subsección \ref{DescripPantallaDetalleMaceración}); y \textbf{ Eliminar maceración}, que elimina todos los datos relacionados a la maceración de la base de datos. Con el fin de llevar a cabo esta última, eliminación, se instancia una clase \textit{SQLiteDatabase}. La cual, luego de obtener la lista de experimentos relacionados al ID de maceración correspondiente, elimina las entradas en la tabla \textit{SensedValues} que se correspondan con la lista de experimentos, luego elimina estos experimentos y, por último, las entradas en las tablas \textit{Grano}, \textit{Intervalo} y \textit{Maceracion} que se correspondan con el ID de maceración. Una vez realizada esta tarea, se cierra la pantalla actual y se accede a la Pantalla Principal (subsección \ref{DescripPantallaPrincipal}).
                
                \par En el diagrama \ref{fig:DiagClaseExperimentActivity} ubicado en el Anexo, pueden observarse las clases y funciones utilizadas para el despliegue de esta pantalla.
        %---------------FIN Descripción ExperimentActivity ---------------
            
        \subsection{Pantalla de detalle de maceración}
        \label{DescripPantallaDetalleMaceración}
            \subsubsection{Descripción}
            Es la pantalla donde se presentan los datos ingresados para esta maceración en la planificación, junto a otros datos calculados.
            En la misma puede ser visualizarse en la parte superior el \textit{ActionBar} donde se indica el nombre de la Maceración. Luego, en el espacio restante se presentan los siguientes valores: tipo de maceración; volumen de mosto; densidad específica deseada; información de los granos a utilizar (nombre, cantidad teórica y, en caso de haber realizado mas de tres experimentos, la cantidad ajustada); e información de los intervalos (duración, temperatura y pH deseados con sus respectivas tolerancias de desvío, volumen y temperatura del agua a ser incorporada, y en caso de maceración por decocción, la temperatura del segundo macerador).
             \begin{figure}[h]
                    \centering
                    \includegraphics[scale=0.2]{software/ScreenCapture/MashDetailActvity.jpg}
                    \caption{Captura de la pantalla de detalle de la maceración}
                    \label{fig:CapturaDetMashAct}
                \end{figure}
            \subsubsection{Detalle de implementación}
            \par Esta pantalla reutiliza el \textit{Activity} \textit{PlanningActivty}, de manera que una vez cargados los valores obtenidos los \textit{widgets} son bloqueados para que no se modifique su valor. 
            \par Para obtener los valores mostrados en la pantalla, primero se realiza una consulta a la base de datos a partir de una instancia de la clase \textit{SQLiteDatabase} en las tablas \textit{Maceracion}, \textit{Intervalo} y \textit{Grano}, obteniendo los valores correspondiente al ID de maceración. Una vez, obtenidos se cargan los valores de tipo de maceración, densidad y volumen deseado. 
            
            \par Luego, para la lista de granos, se calcula la cantidad de insumos según porcentaje de utilización y extracto potencial. Para esta operación se utiliza la función estática \textit{calcCantInsumoTeoRayDaniels} que implementa la ecuación \ref{EcuacionRayDaniels}, definida en la clase \textit{Calculos}. Para la primera aproximación utiliza el valor de rendimiento de equipo equivalente a 70\%. En caso que esta maceración cuente con mas de tres experimentos realizados, también se calcula la cantidad de insumos pero con el valor de rendimiento obtenido a partir de las densidades de los experimentos. El \textit{TextView} del ítem del \textit{ListView} que muestra la descripción del grano, siempre es cargado con el nombre del grano y el primer cálculo. En caso de que se cumple la restricción de cantidad de experimentos, incorporará en la cadena de texto el segundo valor.
            
            \par En el caso de los intervalos, de acuerdo al análisis temático, se calcula el volumen y temperatura de agua que se debe incorporar en cada uno de ellos:
            \begin{itemize}
                \item Simple: Se utiliza la función \textit{temperaturaAguaInicial} presente en la clase \textit{Calculos}, para obtener el volumen y temperatura del agua a agregar para que la infusión alcance la temperatura planificada para el único intervalo. La ecuación se implementó de acuerdo a la ecuación \ref{EcuacionInfusion}.
                
                \item Escalonada: En este caso se computa para cada intervalo el volumen y temperatura que se debe agregar. Para esto se realiza el cálculo basado en las ecuaciones \ref{EcuacionEscalonadaPi} y \ref{EcuacionEscalonadaGamma}. Las funciones implementadas en la clase \textit{Calculos} para este fin poseen los nombres \textit{cantAguaEscalon} y \textit{cantAguaPrimerEscalon} respectivamente.
                
                \item Decocción: Para este tipo de maceración se calcula la cantidad de templa a retirar según la ecuación \ref{EcuacionDecoccion}. Dicho valor es obtenido al llamar la función \textit{cantMostoRetirarDecoccion} de la clase \textit{Calculos}.
            \end{itemize}
            
            \par En el diagrama \ref{fig:DiagClasePlanningActivity} ubicado en el Anexo, pueden observarse las clases y funciones utilizadas para el despliegue de esta pantalla.
        %---------------FIN Descripción InfoMash PlanningActivity ---------------
            
        \subsection{Pantalla de información estadística e histórica}
        \label{DescripPantallaEstadística}
            \subsubsection{Descripción}
            \par En esta pantalla puede ser visualizado un resumen de los experimentos llevados a cabo para la correspondiente maceración.
            \par En la figura \ref{fig:CapturaGeneralFrag-P1} puede visualizarse un \textit{ActionBar} en la parte superior con el nombre de la maceración. Debajo, dos botones tipo pestañas que permiten acceder a los dos paneles,  \textbf{Resumen general} (Figura \ref{fig:CapturaGeneralFrag-P1}) y \textbf{Detalle de temperatura por experimento} (Figura \ref{fig:CapturaExpFrag}). 
            
            \paragraph{Resumen general:} En este panel en primer lugar se ubica un botón que permite alternar entre los dos tipos de gráficas comparativas de la evolución promedio de cada variable (gráfica de líneas y gráfico estadístico descriptivo \textit{Boxplot}\footnote{También conocido como gráfico de caja y bigote (representa valores extremos y el 1.\textsuperscript{er}, 2.\textsuperscript{o} y 3.\textsuperscript{er} cuartil)}). Luego de este, se presentan las siguientes gráficas: Temperatura; pH; y Activación de enzimas. Finalmente, se presenta una serie de valores calculados a partir de los mismos valores que conforman las gráficas, estos son: Número de experimentos válidos; Rendimiento del equipo\footnote{Este valor comienza a ajustarse a partir del tercer experimento válido de una maceración}; Cálculo Teórico de insumos con rendimiento no ajustado \footnote{El ajuste se realiza tomando como rendimiento del equipo el valor de rendimiento calculado, en lugar del estándar 70\%}; Cálculo Teórico de insumos con rendimiento ajustado; por último el calculo práctico de cantidad de insumos. (Figuras \ref{fig:CapturaGeneralFrag-P1} y \ref{fig:CapturaGeneralFrag-P2}) 
            
            \paragraph{Detalle de temperatura por experimento:} En este puede encontrarse en el borde superior un menú desplegable, el cual permite seleccionar el tipo de gráficas a ser comparadas luego, pudiendo ser de cada sensor o del promedio de sensores para cada muestra. De esta manera permite compara el/los mismo/s sensores entre diferentes experimentos. Se presentan algunas de las gráficas por experimento antes mencionadas (Figura \ref{fig:CapturaExpFrag})
            
            \begin{figure}[h]
                \centering
                \includegraphics[scale=0.2]{software/ScreenCapture/GeneralStatistics.jpg}
                \caption{Captura de pantalla de la pantalla de Estadísticas generales - parte 1}
                \label{fig:CapturaGeneralFrag-P1}
            \end{figure}
            
            \begin{figure}[h]
                \centering
                \includegraphics[scale=0.2]{software/ScreenCapture/GeneralStatistics-p2.jpg}
                \caption{Captura de pantalla de la pantalla de Estadísticas generales - parte 2}
                \label{fig:CapturaGeneralFrag-P2}
            \end{figure}
            
            \begin{figure}[h]
                \centering
                \includegraphics[scale=0.2]{software/ScreenCapture/ExperimentStatistics.jpg}
                \caption{Captura de pantalla de la pantalla de Estadísticas de Experimentos}
                \label{fig:CapturaExpFrag}
            \end{figure}
            \subsubsection{Funciones}
            \begin{itemize}
                \item \textbf{Visualización de gráficas de variables:} En el resumen general, se presentan gráficas de los valores promedios de las variables antes mencionadas respecto al tiempo. Se consideran para estas solo los experimentos válidos\footnote{Se consideran válidos aquellos experimentos, que cumplan con la cantidad de mediciones e incluyan el valor de densidad obtenida.} de esta maceración.
            \end{itemize}
            
            \subsubsection{Detalle de implementación}
            \par La obtención de los datos de la maceración se realizan a través de una consulta a la base de datos. Esta filtra los datos de las tabla Experimento con el ID de maceración obtenido de la Pantalla de gestión de maceración (subsección \ref{DescripPantallaGestiónMaceración}) y luego obtiene la totalidad de datos de la tabla \textit{SensedValues} a partir ID de experimentos obtenidos.
            
            \par En el \textit{Fragment} \textbf{Resumen general} se muestran tres gráficas de evolución temporal: temperatura promedio de todos los sensores, pH y activación de enzimas. Este último, es calculado a partir de los dos anteriores utilizando las funciones de activación de enzimas (\textit{alphaAmylase}, \textit{betaAmylase}, \textit{protease} y \textit{betaGlucanase}) implementadas en la clase \textit{Calculos}. La forma en la que estas gráficas presentan los datos puede ser cambiado entre gráfico de lineas (\textit{LineChart} con valor promedio de todos los experimentos) o \textit{boxplot} (\textit{CandleStick}, cargado con los valores mínimo, máximo, primer cuartil y tercer cuartil y \textit{LineChart}, con el valor del segundo cuartil o mediana). (Figura \ref{fig:CapturaGeneralFrag-P1})
            
            \par Debajo de las tres gráficas se muestra un cuadro que contiene el rendimiento del equipo. Este es obtenido a partir del promedio de rendimiento de cada experimento calculado con la ecuación \ref{EcuacionRendimientoMaceracion} implementada en la clase \textit{Calculos}. Luego, se listan los granos utilizados con un \textit{ListView}, los cuales se cargan utilizando una cadena de texto con la cantidad de insumo correspondiente a cada tipo: Teórico (\textit{calcCantInsumoTeoRayDaniels} con valor rendimiento de equipo de 70\%), Ajustado (\textit{calcCantInsumoTeoRayDaniels} con valor rendimiento de equipo calculado) y Real (llamada a la función \textit{calcInsumoReal} de la clase \textit{Calculos} que implementa la ecuación  \ref{EcuacionCantidadGranoEmpirico} con los valores de porcentaje de utilización de cada grano y el valor de rendimiento del equipo). Por último, se muestra la cantidad de experimentos válidos realizados. (Figura \ref{fig:CapturaGeneralFrag-P2})
            
            \par En el \textit{Fragment} \textbf{Detalle de temperatura por experimento} se implementa un \textit{LineChart} por cada uno de los experimentos. Para cada uno de ellos se puede seleccionar distintas opciones de gráfico de temperatura, listadas mediante un \textit{Spinner} en la parte superior del layout del \textit{Fragment}.  (Figura \ref{fig:CapturaExpFrag})
            
            
            \par En el diagrama \ref{fig:DiagClaseMashExpHistoryActivity} ubicado en el Anexo, pueden observarse las clases y funciones utilizadas para el despliegue de esta pantalla.
        %---------------FIN Descripción MashExpHistoryActivity ---------------
        
        \subsection{Pantalla de resultados de experimento}
        \label{DescripPantallaResultadosExperimento}
            \subsubsection{Descripción}
            En esta pantalla puede ser visualizado un resumen de los resultados obtenidos en este experimento.
            Como puede verse en la figura \ref{fig:CapturaDetExpAct} en el borde superior se ubica un \textit{ActionBar} con el la fecha y horario del experimento, debajo se presentan los valores temperatura ambiental, densidad específica del mosto y rendimiento obtenido. En forma posterior, se ubican representaciones gráficas de la evolución temporal de las variables temperatura, pH y activación de enzimas para este experimento, seguidas de la evolución temporal de la temperatura de cada sensor.
            
            \begin{figure}[h]
                \centering
                \includegraphics[scale=0.2]{software/ScreenCapture/DetailExperimentActivity.jpg}
                \caption{Captura de la pantalla con el detalle de un Experimento}
                \label{fig:CapturaDetExpAct}
            \end{figure}
            
            \subsubsection{Detalle de implementación}
            
            \par Con el objetivo de obtener los datos correspondientes al experimento, se instancia la clase \textit{SQLiteDatabase} y se realiza una consulta \textit{SELECT} en la tabla \textit{SensedValues}. Esta última retorna todas las entradas que tenga el ID del experimento obtenido de la Pantalla de gestión de maceración (subsección \ref{DescripPantallaGestiónMaceración}).
            
            \par En la parte superior del layout se muestran tres \textit{Textviews} que contienen los valores de densidad obtenida del experimento, el rendimiento de este y el valor promedio de temperatura ambiental a lo largo del experimento. 
            
            \par Debajo se encuentran siete \textit{LineCharts} que contienen la evolución temporal a lo largo del experimento de: temperatura promedio, pH, activación de enzimas, sensor temperatura N°1, sensor temperatura N°2, sensor temperatura N°3, sensor temperatura N°4. Estos valores son obtenidos a partir del conjunto de valores obtenidos de la consulta de base de datos.
            
            \par En el diagrama \ref{fig:DiagClaseDetailExperimentActivity} y ubicado en el Anexo, pueden observarse las clases y funciones utilizadas para el despliegue de esta pantalla.
        %---------------FIN Descripción DetailExperimentActivity ---------------
        
        \subsection{Pantalla de monitoreo de experimento}
        \label{DescripPantallaMonitoreoExperimento}
            \subsubsection{Descripción}
            Esta pantalla es la encargada de asistir al usuario con el experimento en curso.
            En la figura \ref{fig:CapturaMeasureFrag} puede visualizarse un \textit{ActionBar} en la parte superior con el nombre de la maceración, seguido de dos botones, el primero para cancelar el experimento en curso y el segundo para confirmar la conclusión del experimento. Debajo, dos botones tipo pestañas (\textit{TabLayout}) que permiten acceder a los dos paneles (\textit{Fragments}): \textbf{Mediciones} (Figura \ref{fig:CapturaMeasureFrag}) y \textbf{Etapas} (Figura \ref{fig:CapturaStageFrag}).
            
            \paragraph{Mediciones:}
            En este panel se visualizan los valores obtenidos a través de la estación de recolección de datos para el experimento en curso. Además, enseña dentro de tarjetas (\textit{CardViews}) el porcentaje de avance en el que se encuentre la maceración, los valores planificados para dicho momento y el correspondiente desvío con los valores recolectados.
            \paragraph{Etapas:}
            En este panel se visualiza un resumen de las etapas que componen la maceración, adicionando temporizadores correspondientes al tiempo restante para el inicio de cada una de las mismas.
            
            \begin{figure}[h]
                \centering
                \includegraphics[scale=0.2]{software/ScreenCapture/MeasureFragment.jpg}
                \caption{Captura de pantalla de Monitoreo de Experimento en Curso}
                \label{fig:CapturaMeasureFrag}
            \end{figure}
            \begin{figure}[h]
                \centering
                \includegraphics[scale=0.2]{software/ScreenCapture/StageFragment.jpg}
                \caption{Captura de pantalla de Monitoreo de Etapa en Curso}
                \label{fig:CapturaStageFrag}
            \end{figure}
            
            \subsubsection{Funciones}
            
                \begin{itemize}
                    \item \textbf {Configuración de medición de temperatura:} Manteniendo presionada la tarjeta perteneciente a la temperatura, se abre un dialogo de selección, el cual permite elegir los métodos de representación de los valores de temperatura para monitoreo. Estos pueden ser: promedio, mediana, promedio de valores extremos, y finalmente permite seleccionar los sensores incorporados o exceptuados del cálculo.
                    
                    \item \textbf{Alerta:} En caso de obtenerse un valor fuera de la tolerancia planificada, se muestra una notificación en el dispositivo correspondiente a la variable fuera de rango. 
                    
                    \item \textbf{Finalizar experimento:} Al seleccionar esta opción, en caso que se hayan realizado todas las mediciones del experimento, se muestra un dialogo para que el usuario ingrese el valor de densidad especifica del mosto obtenida. En caso de no haber completado las mediciones, se enseña un mensaje indicando dicha situación.
            
                    \item \textbf{Cancelar experimento:} Esta función permite abortar el experimento en curso. Una vez cancelado, se procede a la eliminación de todos los datos relacionados al experimento cancelado.
                \end{itemize}
            
            \subsubsection{Detalle de implementación}
             \par El \textit{Activity} inicia con el ID de experimento recibido como parámetro de la Pantalla de gestión de maceración (subsección \ref{DescripPantallaGestiónMaceración}), en primer lugar se instancia la clase \textit{SQLiteDatabase} y se realiza una consulta \textit{SELECT} en la tabla \textit{Maceracion} para obtener los valores \textit{intervaloMedTemp} e \textit{intervaloMedPh} y en la tabla \textit{Intervalo}, con la que se obtiene la duración total de la maceración como sumatoria de las duraciones de todos los intervalos de la maceración correspondiente. Luego, con estos datos se realiza una llamada a la API ``ApiEscribe.php''(función \textit{postExperiment}, incluida en la interfaz API) mediante el uso de la librería \textit{Retrofit}. Esta llamada, inicia un nuevo experimento de maceración en la estación de recolección de datos.
             
             \par Con el objetivo de obtener los valores presentes en la base de datos de la estación, se creo una clase \textit{MyWorker} que implementa la librería \textit{WorkManager}. Con la misma, se crea un hilo de ejecución paralelo en el cual se realizan reiteradas llamadas a la API \textit{getSensedValues} (la que invoca a la API ``ApiGet.php'') con el fin de obtener nuevos valores recolectados para luego ser insertados en la base de datos de la aplicación. Estas peticiones concluyen luego de un numero de iteraciones separadas por un lapso de tiempo. El lapso es equivalente a la mitad del valor especificado para el intervalo de medición de temperatura, y el número de iteraciones se obtiene al dividir la duración total del experimento sobre este lapso de separación.
             
            \par En este \textit{Activity} se encuentran dos pestañas: Mediciones(\textit{MeasureFragment}) y Etapas(\textit{StageFragment}). En la primera se encuentran cargados en \textit{CardViews} los valores del último \textit{SensedValues} insertado en la base de datos de experimento en curso y en la segunda se muestra información relevantes a las etapas.
            
            \par Debido a que la clase \textit{MyWorker} no esta habilitada para modificar la interfaz gráfica, se implementó dentro del \textit{MeasureFragment} un subhilo de ejecución. Este último se encarga de realizar consultas \textit{SELECT} a través de una instancia de \textit{SQLiteDatabase} con periodo de repetición equivalente al utilizado en \textit{MyWorker}. Mediante estas consultas, son obtenidos  los valores de la tabla \textit{SensedValues} mas actuales para así actualizar los \textit{CardViews} con la correspondiente información.
            
            \par En forma adicional a los valores obtenidos de la estación de recolección, se visualiza el promedio de temperatura del macerador, la activación de enzimas y el porcentaje de avance del experimento. 
            \begin{itemize}
                \item El primer valor puede ser configurado para que ignore un/algunos valor/es de un/varios sensor/es y para que el método de cálculo de promedio sea media aritmética, mediana o promedio de valores extremos. Con la finalidad de implementar estas configuraciones, se utilizo un I donde el usuario puede visualizar y seleccionar que sensor esta habilitado y alguno de los tres tipos de cálculos, implementados en la clase \textit{Calculos}. Esta configuración es almacenada en el \textit{SharedPreferences} ``ConfTemp'' y es cargada cada vez que se inicia una nueva experiencia de medición. 
                
                \item Las activaciones de cada tipo de enzimas son calculadas a partir del valor promedio de temperatura y el pH actual (el último valor recolectado disponible). Para ello se utiliza las funciones \textit{alfaAmilasa}, \textit{betaAmilasa}, \textit{proteasa} y \textit{betaGlucanasa}, implementadas en la clase \textit{Calculos}. 
                
                \item El porcentaje de avance se obtiene a partir de la división de la cantidad de valores insertados en la tabla SensedValues para el experimento en curso y la cantidad de mediciones total del mismo. Adicionalmente, se visualiza la etapa actual que se esta sensando, valor obtenido a partir de la comparación de la cantidad de valores sensados y la acumulación de mediciones correspondiente a cada intervalo de la maceración. Por último, se incorpora un cronómetro (widget \textit{Chronometer}) que informa el tiempo transcurrido desde el inicio del experimento.
            \end{itemize}
            
            \par En el segundo \textit{Fragment} del \textit{CurrentExperienceActivity}, se incluye un \textit{RecyclerView} con una lista de \textit{CardViews} por cada etapa de medición que posea la maceración. En cada uno de estos, se pueden ver los valores de la planificación como son duración, temperatura y pH deseados con sus correspondientes desvíos tolerados, y el volumen y temperatura de agua a incorporar en el intervalo (Maceración Escalonada) o cantidad de mosto a retirar (Maceración por Decocción). Adicionalmente, se incorpora un \textit{Chronometer} en modo temporizador para cada intervalo que indica cuanto tiempo resta para que comience dicha etapa.
            
            \par Por último, en el \textit{ActionBar} del \textit{Activity} se incorporaron dos opciones: Cancelar Experimento y Finalizar Experimento. Al seleccionar la primer opción, inicialmente se cierra el Activity y se inicia la Pantalla de gestión de maceración (subsección \ref{DescripPantallaGestiónMaceración}) adjuntando el ID de maceración actual. Luego, se realiza una llamada a la API ``apiCancel.php'' a través de la función \textit{cancelExperiment} implementada en la \textit{interface} de \textit{RetroFit}. Una vez finalizado este proceso, se enseña un cartel emergente indicando un resultado satisfactorio.
            
            \par Al presionar la opción Finalizar Experimento, se pueden presentar dos escenarios: El primero ocurre cuando no se obtuvieron aún todas las mediciones y se le indica al usuario que no puede finalizar la medición. El segundo, ya finalizado el experimento, se muestra un \textit{AlertDialog} con un \textit{EditText} que le permite al usuario ingresar la densidad específica del mosto resultante. Una vez finalizada esta operación, se retorna a la Pantalla de gestión de maceración (subsección \ref{DescripPantallaGestiónMaceración}) correspondiente al ID de maceración actual.
            
            \par En los diagramas \ref{fig:DiagClaseCurrentExperienceActivityP1} y \ref{fig:DiagClaseCurrentExperienceActivityP2} ubicados en el Anexo, pueden observarse las clases y funciones utilizadas para el despliegue de esta pantalla.
            %---------------FIN Descripción CurrentExperimentActivity ---------------
            
    \subsection{Código fuente}
    \par El código implementado para esta aplicación se encuentra alojado en el repositorio de GitHub \url{https://github.com/damianpna88/proyectofinal} dentro de la carpeta APP. Para compilar el proyecto, es requerida una versión actualizada de Android Studio con la API 24 (Nougat 7.0) descargada.