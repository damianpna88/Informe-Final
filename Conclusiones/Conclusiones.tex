\chapter{Conclusiones y trabajos futuros}
\par
En este último capítulo se describen las conclusiones finales obtenidas a partir de las pruebas y de un análisis de las funcionalidades provistas por el sistema. En forma adicional, se presentan una serie de propuestas para trabajos futuros que permitirían mejorar el alcance y rendimiento del sistema.

%podria ser conclusiones y resultados?
\section{Conclusiones} % Cumplimos lo que nos propusimos? era cierto lo que se dijo?

\par En principio se consideran necesarios de destacar los ejes que dominan la dinámica de todo proyecto: la triple restricción ``Alcance, Tiempo y Costo''. El primero, fue establecido en etapas tempranas del desarrollo del anteproyecto, y no reviste la condición de ser reducido o aumentado. Por ende, esta restricción fue la dominante en la dinámica de trabajo. En segundo lugar, respecto al tiempo, se resalta una inicial sub-estimación de tiempo-esfuerzo producto de la inexperiencia de los proyectistas en las áreas de conocimiento que atravesaron este proyecto. Es así, que las tareas de aprendizaje y desarrollo resultaron en una extensión de tiempo substancialmente superior. Finalmente, en consecuencia de las dos restricciones anteriores, el costo de los recursos humanos aumentó de manera significativa conforme lo hizo el tiempo.

\par En este proyecto se planteó el desarrollo de un prototipo de hardware y software para asistir al productor de cerveza artesanal en la planificación, seguimiento y evaluación del proceso maceración. Tomando como guía los objetivos planteados para el mismo (Sección \ref{secccionObjetivos}), se fueron desempeñando las tareas correspondientes para el cumplimiento de los mismos. A continuación son evaluados estos objetivos:

\subsubsection{Asistencia para planificación} 
\par A raíz de las pruebas realizadas (sección \ref{CapituloPruebas}), fue comprobado que el sistema contribuye en la asistencia del proceso de maceración de forma práctica y útil. A partir de ciertos parámetros básicos, el sistema facilita los pasos e insumos necesarios para llevar a cabo la maceración deseada. Luego, los valores de estimación obtenidos en la Pantalla de detalle de maceración \ref{DescripPantallaDetalleMaceración} cumplen el requerimiento de auto-ajuste conforme son llevados a cabo una mayor cantidad de experimentos. En forma consecuente, se optimiza la cantidad de insumos utilizada y se asiste al productor asegurando el cumplimiento de los objetivos planteados para la receta. Además, incorpora en este apartado herramientas para llevar a cabo de forma mas sencilla maceraciones complejas (que requieran de dos o más etapas). Por ende, puede afirmarse que el sistema cumple el objetivo de ser una herramienta de útil en la planificación de maceraciones.

\subsubsection{Asistencia para seguimiento}
\par La pantalla de monitoreo de experimento \ref{DescripPantallaMonitoreoExperimento} permite el seguimiento durante el avance de una maceración a través de los múltiples datos allí presentados. En este apartado, se considera importante de mencionar un planteo inicial en cuanto al intervalo de medición de temperatura, el cual pudiese tener una duración tan reducida como la que el usuario requiera. Luego, esta decisión de diseño se vio restricta por limitaciones funcionales del hardware. El mismo requiere de un tiempo aproximado de 30 segundos para recolectar e insertar en la base de datos los valores obtenidos por los sensores. Finalmente, tanto el intervalo de medición de pH (con su mínimo de dos minutos de estabilización de medición) como el de temperatura cuentan con un límite inferior. No obstante a esto, estas limitaciones temporales no perjudican al funcionamiento del sistema debido a que los cambios en estas variables, en este contexto, aun permiten al usuario reconocer la desviación y realizar una corrección en consecuencia. De esta forma, aun contando con los límites inferiores en el intervalo de tiempo de medición, el seguimiento del proceso se realiza de forma correcta.

\subsubsection{Asistencia para la evaluación}  
\par La utilización de gráficas en la aplicación produjo resultados satisfactorios debido a su utilidad en la identificación y comprensión rápida del comportamiento de las variables intervinientes. Estas gráficas, pueden presentar datos puntuales de una medición de un experimento hasta un resumen de todas las experiencias realizadas, permitiendo así un análisis específico o del conjunto de procesos.
\par Por otro lado, el sistema proporciona el valor de rendimiento del equipo para cada experimento de maceración, luego en conjunto a las gráficas es posible realizar un análisis de las dinámicas intervinientes en la producción de un determinado resultado, pudiendo así repetir o perfeccionar una receta.
Es de comentar el cumplimiento de una de las finalidades expuestas en el anteproyecto, permitir identificar dinámicas de comportamiento térmico dentro del recipiente de macerado. El sistema a partir de la integración de múltiples sensores de temperatura cumple este objetivo. Aún mas, para la realización de las pruebas empíricas pudo identificarse que el contenedor dentro del que se realizaban las maceraciones perdía temperatura por varios sectores del macerador, afectando negativamente a los resultados del proceso. A partir de los resultados arrojados por este sistema, fueron implementadas mejoras al recipiente de manera de corregir esta deficiencia.
Finalmente, se considera el conjunto de herramientas como útiles para el productor, cumpliendo con el objetivo. 

\section{Trabajos futuros} %A delirar fuerte.
\par En esta sección se enumeran una serie de mejoras en las que podría incurrir este sistema/proyecto con la finalidad de aumentar los beneficios otorgados por el mismo.

\begin{itemize}
    \item Optimización de código: la optimización de código puede conllevar la mejora del aprovechamiento de los recursos y la reducción de tiempos de respuesta del sistema. En particular, la implementación del prototipo posee una demora aproximada de 50 segundos para realizar la medición de todos los sensores. Esto conllevó a la restricción de no poder utilizar una período de medición inferior a un minuto. Aplicar una optimización en el código favorecería al aumento de la frecuencia de medición permitida.
    
    \item Mejora de componentes utilizados: La exactitud y precisión de los sensores estuvieron limitados al presupuesto destinado por parte de los proyectistas. En particular, el sensor de pH utilizado produce mediciones inestables a lo largo del tiempo conllevando a una disminución de la exactitud de sus medidas. El empleo de un sensor de mejor calidad produciría el mejoramiento de la exactitud de las medidas, ofreciendo mayor confianza en los resultados.
    
    \item Integración con otros sistemas: El mayor beneficio para el productor lo obtendría a partir de un sistema que lo asista durante todo el proceso de producción de cerveza. Con una herramienta de estas características, cada variable que interviene puede ser controlada y ajustada para que el producto final tenga las propiedades deseadas. Ya que el prototipo propuesto es limitado al proceso de maceración, la integración del mismo con otros sistemas que contribuyan con la asistencia de los restantes procesos de producción otorgaría un mayor beneficio para el usuario del mismo.
    
    \item Monitoreo de múltiples maceraciones: Este prototipo fue planteado para llevar el seguimiento de un único equipo para macerar. Si el productor cuenta con mayor cantidad de equipos destinados a este proceso para paralelizar su producción, este sistema no tendría capacidad de asistirlo en cada uno de ellos. Para estos casos, tendría mayor beneficio que el modulo de hardware actué en forma independiente y se implemente una aplicación que pueda obtener valores de cada uno de estos de forma simultánea.
    
    \item Instrucciones de corrección del proceso en tiempo real: Cuando se incurre en un desvío de las variables temperatura o pH, existen métodos para corregir las mismas al valor deseado. Esto requiere un mayor conocimiento de física (temperatura, termodinámica) y química (pH, ácidos y bases) por parte del productor, siendo obviados usualmente por métodos empíricos utilizados en el home brewing. El productor obtendría mayor beneficio del asistente si este le indicará durante el proceso los pasos a seguir para restablecer la templa a los valores deseados.
    
    \item Incorporación de densímetro digital: Como se menciono a lo largo de este documento, la densidad es una variable de interés al finalizar la maceración. El sistema desarrollado requiere la inserción manual de dicho valor por parte del productor, conllevando al mismo a la realización de la medición. Esta última puede incurrir en una modificación del valor real, si el usuario no ingresa correctamente el valor o no es preciso al momento de realizar el proceso, ya que el sistema no lo detectará. La incorporación de un componente que se encargue de realizar esta tarea reduciría el esfuerzo del productor y aumentaría la confianza del valor obtenido. 
    
    \item Base de datos en la nube: El diseño del sistema se estableció a partir de la ubicación de dos bases de datos, en la estación de recolección de datos y en la aplicación móvil. A partir de este diseño fue posible cumplir con los objetivos del proyecto. Sin embargo, en lugar de establecer la base de datos en la aplicación sino en un servidor externo (servidor web), permitiría utilizar las herramientas provistas por la aplicación en mas de un dispositivo móvil, así como también en un navegador web.
    
    \item Circuitería impresa como PCB\footnote{“placa de circuito impreso”, superficie constituida por caminos, pistas o buses de material conductor laminadas sobre una base no conductora.}: En cuanto a las conexiones de los sensores con las placas electrónicas, las mismas fueron realizadas mediante la utilización de cables y una protoboard. Esta configuración aunque eficaz es poco eficiente, ya que dificulta su transporte por la relativa sencillez con la que los mismos pueden desconectarse. Imprimir estas conexiones sobre una placa impresa resultaría en un valor agregado al sistema.
    
    \item Facilidades de configuración WiFi\textsuperscript{\textregistered} de la estación de recolección de datos: En los alcances de este proyecto fue planteado que la configuración WiFi\textsuperscript{\textregistered} de la estación de recolección de datos sería realizada en forma manual. Se plantea como mejora la posibilidad de realizar esta configuración de alguna otra manera, por ejemplo, a partir de una conexión Bluetooth\textsuperscript{\textregistered} y una funcionalidad implementada con esta conexión para este fin.
    
    \item Placa electrónica de diseño propio: Las placas electrónicas aquí utilizadas devinieron en ventajas y limitaciones que debieron ser atendidas y que devinieron en complicaciones para su implementación. Mediante un diseño electrónico propio podría obtenerse un componente que satisfaga directamente los requerimientos necesarios para este sistema, aminorando de esta manera la complejidad en su construcción y mejorando su funcionamiento.
    
    \item Optimización de código Android: la aplicación aquí desarrollada tiene el objetivo de satisfacer sus requerimientos de implementación, no teniéndose en cuenta una optimización en cuanto al uso de recursos del dispositivo. Por ende, una optimización del mismo solventaría esta situación.
    
    
\end{itemize}






