\chapter{Conclusiones y Trabajos futuros}
\par
En este último capítulo se describen las conclusiones finales obtenidas a partir de las pruebas y de un análisis de las funcionalidades provistas por el sistema. Además se presentan una serie de propuestas para trabajos futuros que permitirían mejorar el alcance y rendimiento del sistema.

%podria ser conclusiones y resultados?
\section{Conclusiones} % Cumplimos lo que nos propusimos? era cierto lo que se dijo?
\par En este proyecto fue planteado el desarrollo de un prototipo de hardware y software para asistir al productor de cerveza artesanal en la planificación, seguimiento y evaluación del proceso maceración. Tomando como guía los objetivos planteados para el mismo (Sección \ref{secccionObjetivos}), se fueron desempeñando las tareas correspondientes para el cumplimiento de los mismos. A continuación evalúan los tres ejes del sistema:
%no se si va lo de los objetivos

%evitar caer en beneficios; ya se habló en la sección objetivos..

%Aca podemos hablar de subestimaciones (o que acertamos) que tuvimos también, pero la cosa es que no agregamos cosas del anteproyecto. [Respecto al tiempo y alcance mas que nada. El costo por ahí va a ser un poco engorroso porque no tuvimos un seguimiento fino, aunque algo se podría garabatear.]

\subsubsection{Asistencia en planificación} 
\par A raíz de las pruebas realizadas (sección sumbudrule), se comprobó que el sistema contribuía a la asistencia del proceso de la maceración de forma práctica, útil y precisa. Los valores de estimación obtenidos en la Pantalla de detalle de maceración \ref{DescripPantallaDetalleMaceración} concordaron con un nivel de precisión x (o nivel de error menor) a los obtenidos en los experimentos guía/ejemplo/patrón/de prueba. Por ende se puede afirmar que los cálculos (Rey Daniel) utilizados estiman valores geniales.
%Como trabajaron las enzimas nunca lo vamos a poder saber. Salvo que llevemos a un laboratorio las muestras obtenidas. Si bien nos desvalora el proyecto, podemos hablar de algo de esto.


% estos son beneficios obtenidos
%Las validaciones pertinentes durante el proceso permite que no se comentan errores en la planning. Los cálculos para maceraciones escalonada o decocción reducen el nivel de habilidades par realizar cálculos algebraicos por parte del productor, dándole una facilidad para que utilice estos tipos de maceración menos utilizados en el ámbito artesanal.

\subsubsection{Asistencia en seguimiento} 
\par La pantalla de monitoreo de experimento \ref{DescripPantallaMonitoreoExperimento} permite el seguimiento del avance de una maceración. La frecuencia de medición de la estación de recolección fue precisa, no siendo cumplido este atributo en la visualización de los mismos en la aplicación ya que se observó cierta variación temporal introducida a través de la interfaz (referencia a una prueba hecha). No obstante esto, estas alteraciones temporales en la visualización no alcanzan un valor significativo para considerarse causa de un problema en el seguimiento.

%Aca tambien podemos decir que no pudimos medir a menos de 40 segundos por la limitante del jargüart
% Podemos mencionar que usar múltiples sensores dió sus frutos o no. No se si ponerlo en monitoreo o evaluación

\subsubsection{Asistencia en evaluación}  
%No se que mierda poner aca.
\par La utilización de las gráficas implementadas en la aplicación con la librería MPAndroidChart produjeron resultados satisfactorios, ya que se le permite al usuario el análisis gráfico puntal de una medición de un experimento hasta un resumen de todas las experiencias realizadas.
\par El informe del valor preciso de rendimiento del equipo para cierta maceración permite una comparación simple y útil entre las distintas formas que puede realizarse la misma para una misma receta. 


\section{Trabajos futuros} %A delirar fuerte.

% Probar en un macerador mas grande, ya que nadie produce de a 2 litros.
% unir a sistema que controle todo el proceso de producción de cerveza.
% utilizar circuito propio en lugar de Rasp + ArduinoUNO?
%intrucciones agrega x cantidad de acido porque tu ph esta muy alto, agrega bicarbonato de sodio por tenes un pH muy bajo, clavate una empanada salteña de carne cortada a cuchillo, si usara una hornalla para calentar que ponga a calentar que se le esta enfriando la templa (el guiso), 


MSToolKit puede ser el puntapié inicial para un proyecto mucho más grande en tamaño y alcance. Como se mencionó anteriormente, los productos pagos brindan muchas funcionalidades y características ausentes en los productos libres. Como trabajos a un futuro cercano se propone la realización de:
\begin{itemize}
	\item Soporte para JSON: En la última versión de ModSecurity, mediante la nueva directiva \textit{SecAuditLogFormat} está la posibilidad de establecer el formato JSON o nativo para los archivos de auditoría. Actualmente todo el tratamiento de estos registros se realiza solo con el formato nativo, la propuesta es realizar el trabajo necesario para que MSToolKit también soporte el formato JSON.
	\item Precarga de directivas y posibles valores: MSToolKit actualmente trae precargadas las directivas básicas que vienen explicitas por defecto en los archivos de configuración de ModSecurity y de MLogC. La propuesta es agregar todas las directivas posibles de estos archivos, facilitándole al usuario las opciones disponibles a la hora de establecer la configuración deseada.
	\item Precarga de reglas: MSToolKit actualmente carga las reglas a medida que las mismas se disparan en algún evento. Esto permite utilizar varias versiones del Core Rule Set sin tener inconvenientes, pero por otro lado las reglas que nunca fueron disparadas quedan desconocidas para la aplicación. Además, para el módulo de activación y desactivación de reglas individuales sería bueno contar con una lista completa y actualizada de las reglas disponibles para el firewall. Para esto se propone la realización de funcionalidades que recorran el directorio de las reglas y las vayan almacenando en la base de datos.
\end{itemize}

Como trabajos a mediano plazo, se propone:
\begin{itemize}
	\item Validaciones de configuración: MSToolKit brinda la posibilidad de agregar las directivas y sus respectivos valores. Sin embargo, existen directivas que podrían dejar sin efecto a otras dependiendo del orden en el cual son escritas, o si por error u omisión se establecieran valores incorrectos, queda como responsabilidad del administrador del sistema la detección de dicho error. Por ejemplo, si se establece la directiva \textit{SecAuditLogType Concurrent} y no se establece la ubicación del binario de MLogC, ModSecurity registrará los archivos en carpetas y no los enviará a MSToolKit, pues no pudo encontrar a MLogC. Otro caso sería escribir incorrectamente el valor de una directiva, por ejemplo, establecer \textit{SecAuditLogTipe} (notar que se cambió la letra 'y' por la 'i'), lo que resulta en un error al querer recargar o iniciar el servidor. Para evitar este tipo de inconvenientes se propone desarrollar las funcionalidades de verificación necesarias para todas las combinaciones posibles entre las directivas y alertar aquellos casos que no resulten en un error pero se inhabiliten otras opciones.
	\item Incorporar el proyecto a Docker: para facilitar la instalación y configuración del firewall, se propone crear una imagen Docker que contenga los elementos básicos para montar el servidor de aplicaciones con ModSecurity y MSToolKit ya instalados. De esta forma, quien haga uso del contenedor, tendría que añadir su aplicación al mismo, brindándole portabilidad y seguridad en un solo lugar.
	\item Georeferenciación de las IP: para los eventos detectados y cuando la georeferenciación esté activada, se propone desarrollar un módulo que contenga un mapa que describa las ubicaciones de las direcciones IP que dispararon los eventos, además de mostrar aquellas actualmente bloqueadas. Además se brindarán opciones para bloquear automáticamente las IP expuestas y asociarlas al historial de eventos de la misma.
\end{itemize}