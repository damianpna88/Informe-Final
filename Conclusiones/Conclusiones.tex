\chapter{Conclusiones y Trabajos futuros}
\par
En este último capítulo se describen las conclusiones finales obtenidas a partir de las pruebas y de un análisis de las funcionalidades provistas por el sistema. Además se presentan una serie de propuestas para trabajos futuros que permitirían mejorar el alcance y rendimiento del sistema.

\section{Conclusiones}
En este proyecto fue planteado un trabajo de investigación sobre los firewalls de aplicación web (WAF) y el desarrollo de una herramienta web que facilite la gestión de ModSecurity. Luego de estudiar las características presentes en los productos dentro del mercado, se realizó un análisis comparativo entre aquellos WAF que cumplían con una serie de requerimientos, siendo ModSecurity el más adecuado para continuar con el trabajo.

\par
Dado que la mayor parte de los programas de gestión de WAF se presentan como aplicaciones de escritorio, también representaba un desafío llevar a cabo el desarrollo con tecnologías web que permitieran una mayor compatibilidad con los dispositivos de hoy en día. Como resultado de esta etapa, fue desarrollado MSToolKit, quien además de cumplir con los requerimientos planteados por la Secretaría de Tecnologías para la Gestión (STG) también pretende ser un aporte a la comunidad de software libre, quedando a disposición de quien desee adquirirla (disponible en GitHub).
\par
Finalmente, en las pruebas realizadas en la STG, si bien no se realizaron pruebas sobre MSToolKit, si se probó el desempeño de ModSecurity, y se estima que desplegar la aplicación desarrollada se podrá realizar sin grandes esfuerzos y sin que ello represente una diferencia significativa en el consumo de recursos. Este es un aspecto muy importante, ya que actualmente la STG está utilizando ModSecurity y se encuentra con dificultades para las cuales está diseñada MSToolKit, por lo que en un paso posterior a este proyecto podría implementarse en un entorno productivo y con un impacto real, lo que le dará visualización y respaldo al proyecto. Además, al no existir un producto de características similares y estar bajo la licencia de software libre Apache 2.0, se espera obtener la colaboración de desarrolladores interesados en añadir funcionalidades al proyecto.

\section{Trabajos futuros}
MSToolKit puede ser el puntapié inicial para un proyecto mucho más grande en tamaño y alcance. Como se mencionó anteriormente, los productos pagos brindan muchas funcionalidades y características ausentes en los productos libres. Como trabajos a un futuro cercano se propone la realización de:
\begin{itemize}
	\item Soporte para JSON: En la última versión de ModSecurity, mediante la nueva directiva \textit{SecAuditLogFormat} está la posibilidad de establecer el formato JSON o nativo para los archivos de auditoría. Actualmente todo el tratamiento de estos registros se realiza solo con el formato nativo, la propuesta es realizar el trabajo necesario para que MSToolKit también soporte el formato JSON.
	\item Precarga de directivas y posibles valores: MSToolKit actualmente trae precargadas las directivas básicas que vienen explicitas por defecto en los archivos de configuración de ModSecurity y de MLogC. La propuesta es agregar todas las directivas posibles de estos archivos, facilitándole al usuario las opciones disponibles a la hora de establecer la configuración deseada.
	\item Precarga de reglas: MSToolKit actualmente carga las reglas a medida que las mismas se disparan en algún evento. Esto permite utilizar varias versiones del Core Rule Set sin tener inconvenientes, pero por otro lado las reglas que nunca fueron disparadas quedan desconocidas para la aplicación. Además, para el módulo de activación y desactivación de reglas individuales sería bueno contar con una lista completa y actualizada de las reglas disponibles para el firewall. Para esto se propone la realización de funcionalidades que recorran el directorio de las reglas y las vayan almacenando en la base de datos.
\end{itemize}

Como trabajos a mediano plazo, se propone:
\begin{itemize}
	\item Validaciones de configuración: MSToolKit brinda la posibilidad de agregar las directivas y sus respectivos valores. Sin embargo, existen directivas que podrían dejar sin efecto a otras dependiendo del orden en el cual son escritas, o si por error u omisión se establecieran valores incorrectos, queda como responsabilidad del administrador del sistema la detección de dicho error. Por ejemplo, si se establece la directiva \textit{SecAuditLogType Concurrent} y no se establece la ubicación del binario de MLogC, ModSecurity registrará los archivos en carpetas y no los enviará a MSToolKit, pues no pudo encontrar a MLogC. Otro caso sería escribir incorrectamente el valor de una directiva, por ejemplo, establecer \textit{SecAuditLogTipe} (notar que se cambió la letra 'y' por la 'i'), lo que resulta en un error al querer recargar o iniciar el servidor. Para evitar este tipo de inconvenientes se propone desarrollar las funcionalidades de verificación necesarias para todas las combinaciones posibles entre las directivas y alertar aquellos casos que no resulten en un error pero se inhabiliten otras opciones.
	\item Incorporar el proyecto a Docker: para facilitar la instalación y configuración del firewall, se propone crear una imagen Docker que contenga los elementos básicos para montar el servidor de aplicaciones con ModSecurity y MSToolKit ya instalados. De esta forma, quien haga uso del contenedor, tendría que añadir su aplicación al mismo, brindándole portabilidad y seguridad en un solo lugar.
	\item Georeferenciación de las IP: para los eventos detectados y cuando la georeferenciación esté activada, se propone desarrollar un módulo que contenga un mapa que describa las ubicaciones de las direcciones IP que dispararon los eventos, además de mostrar aquellas actualmente bloqueadas. Además se brindarán opciones para bloquear automáticamente las IP expuestas y asociarlas al historial de eventos de la misma.
\end{itemize}