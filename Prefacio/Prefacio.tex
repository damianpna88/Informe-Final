\chapter{Prefacio}
\par
En este trabajo se presenta el camino recorrido desde la concepción a las pruebas del presente proyecto, atravesando así su diseño y desarrollo. El mismo es requisito para aprobar la asignatura Proyecto Final de Carrera, y así obtener el título de grado de la carrera de Ingeniería en Informática de la Facultad de Ingeniería y Ciencias Hídricas de la Universidad Nacional del Litoral.
\par
Aquí, se detallan las motivaciones que dieron origen al mismo, las investigaciones realizadas, los aspectos contemplados, tecnologías utilizadas, decisiones de diseño y desarrollo técnico, resultados y conclusiones finales.
\par
Se presenta un informe compuesto por ocho capítulos:
\par
Capítulo 1 - \textit{Introducción}, en este capítulo se describe el marco en el cual se realiza el proyecto, la problemática que pretende resolver y los diversos objetivos planteados.
\par
Capítulo 2 - \textit{Marco temático: Producción de cerveza}, en este capítulo se desarrolla y explica el proceso de fabricación de cerveza artesanal, con especial atención en el subproceso de maceración.
\par
Capítulo 3 - \textit{Ingeniería de requerimientos}, en este capítulo se presentan los requerimientos funcionales y no funcionales, y los casos de uso.
\par
Capítulo 4 - \textit{Componente de hardware}, en este capítulo se analiza el componente electrónico abordando la elección de los elementos que lo componen, el diseño de interconexión, y la descripción del software que se ejecuta dentro del mismo.
\par
Capítulo 5 - \textit{Componente de interfaz Hardware - Software}, en este capítulo se presenta y justifica la tecnología utilizada, el diseño concebido, y finalmente, se describe el componente construido.
\par
Capítulo 6 - \textit{Componente de software} (aplicación móvil), en este capítulo se describen las consideraciones tomadas para la elección de la plataforma sobre la cual se desarrolló la aplicación, el diseño conceptuado, y finalmente, una descripción detallada del desarrollo realizado.
\par
Capítulo 7 - \textit{Pruebas}, en este capítulo se presentan los resultados de diferentes pruebas llevadas a cabo sobre los diferentes aspectos del funcionamiento del sistema.
\par
Capítulo 8 - \textit{Conclusiones}, por último, en este capítulo se exponen las conclusiones del trabajo realizado y se proponen posibles mejoras futuras.
