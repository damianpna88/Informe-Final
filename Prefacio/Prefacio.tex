\chapter*{Prefacio}
\par
En este trabajo se presentará el camino recorrido desde la concepción hasta el diseño y desarrollo del proyecto en cuestión. El mismo, es requisito para aprobar la asignatura Proyecto Final de Carrera, y así obtener el título de grado de la carrera de Ingeniería en Informática de la Facultad de Ingeniería y Ciencias Hídricas de la Universidad Nacional del Litoral.
\par
Aquí, se detallan las motivaciones que dieron origen al mismo, las investigaciones realizadas, los aspectos contemplados, tecnologías utilizadas, decisiones de diseño y desarrollo técnico, resultados y conclusiones finales.
\par
Se presenta un informe compuesto por ocho capítulos:
\par
Capítulo 1 - Introducción, se describe el marco en el cual se realiza el proyecto, la problemática que pretende resolver y los diversos objetivos planteados.
\par
Capítulo 2 - Marco temático: Producción de Cerveza, se describe en detalle el tema dentro del cual se enmarca el sistema desarrollado.
\par
Capítulo 3 - Ingeniería de requerimientos, se presentan los requerimientos funcionales y no funcionales, y los casos de uso.
\par
Capítulo 4 - Componente de Hardware, se presenta un análisis de tecnologías alternativas para la implementación del componente de hardware, su diseño y una descripción del componente construido.
\par
Capítulo 5 - Componente de interfaz Hardware-Software, se presenta un análisis de tecnologías alternativas para la implementación de la interfaz hardware-software, su diseño y una descripción del componente construido.
\par
Capítulo 6 - Componente de Software (aplicación móvil), se presenta un análisis de tecnologías alternativas para la implementación del software, su diseño y una descripción del componente construido.
\par
Capítulo 7 - Pruebas, se presentan los resultados de un análisis exhaustivo realizado al funcionamiento del sistema bajo diferentes pruebas.
\par
Capítulo 8 - Conclusiones, por último, se exponen las conclusiones del trabajo realizado y se proponen posibles mejoras futuras.
