\pagestyle{plain}
\afterpage{\null\newpage}\thispagestyle{plain}
\newpage
\chapter{Resumen}
\par
    El aumento en la producción y consumo de cerveza artesanal a nivel mundial se extiende al territorio argentino con índices de crecimiento del 30 \% anual como menciona \cite{Cuculiansky17}. Es de destacar que en este contexto, el aporte personal y la impronta de cada cervecero le permiten mejorar la calidad de su producto, aumentando su competitividad en este mercado. La apariencia y el sabor definen la calidad de esta bebida fermentada, resultado de la precisión en el control del proceso y el ajuste empírico.
\par
    En este marco, se propuso el desarrollo de un prototipo de hardware y software con el fin de asistir al productor de cerveza artesanal en la planificación, seguimiento y evaluación de una etapa del proceso llamado maceración. En el presente informe, se documentan las fases intervinientes que se abordaron durante la construcción del mismo.
\par
    En forma final, son presentados los resultados obtenidos donde se verifica que el sistema cumple los objetivos planteados. Luego, se presentan las conclusiones obtenidas y se explaya sobre futuros desarrollos que mejorarían y ampliarían el funcionamiento de este prototipo.