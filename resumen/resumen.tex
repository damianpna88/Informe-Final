
\afterpage{\null\newpage}
\newpage
\chapter*{Resumen}
\par
    El aumento en la producción y consumo de cerveza artesanal a nivel mundial se extiende al territorio argentino con índices de crecimiento del 30 \% anual. Es de destacar que en este contexto, la apariencia y el sabor evalúan la calidad de esta bebida fermentada, resultado de la precisión en el control del proceso y el ajuste empírico, otorgándole la impronta al fabricante y permitiéndole competir en el mercado.
\par
    El presente documento establece pautas y fundamentos para el desarrollo de un prototipo de hardware y software para asistir al productor de cerveza artesanal en la planificación, seguimiento y evaluación de una etapa del proceso llamada maceración. Esta etapa, la primera de las siete etapas de producción, es considerada crítica por los resultados y la complejidad que reviste. La misma, consiste en la realización de una infusión con granos malteados, produciendo los subproductos esenciales que marcan el camino para el resto de los procesos, afectando directamente, y en gran medida los resultados planificados, es decir, la cerveza a conseguir.
